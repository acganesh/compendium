\documentclass[12pt]{article}

\usepackage{amsmath}
\usepackage{amssymb}
\usepackage{fancyhdr}
\usepackage{todonotes}
\usepackage{amsthm}
\usepackage{amsopn}
\usepackage{amsfonts}
\usepackage{mathtools}
\usepackage{libertine}

\newtheorem*{theorem}{Theorem}
\newtheorem*{definition}{Definition}
\newtheorem*{remark}{Remark}
\newtheorem*{claim}{Claim}
\newtheorem*{example}{Example}
\newtheorem*{lemma}{Lemma}
\newtheorem*{prop}{Proposition}

\usepackage{latexsym}
\usepackage{bbm}
\usepackage[small,bf]{caption2}
\usepackage{graphics}
\usepackage{epsfig}
\usepackage{amsopn}
\usepackage{url}

\newcommand{\bc}{\binom}
\newcommand{\bx}{\boxed}
\newcommand{\RR}{\mathbb{R}}
\newcommand{\II}{\mathbb{I}}
\newcommand{\Ra}{\mathcal{R}}
\newcommand{\EE}{\mathbb{E}}
\newcommand{\HH}{\mathcal{H}}
\newcommand{\NN}{\mathbb{N}}
\newcommand{\FF}{\mathbb{F}}
\newcommand{\ve}{\varepsilon}
\newcommand{\eps}{\epsilon}
\newcommand{\la}{\langle}
\newcommand{\ra}{\rangle}
\newcommand{\mbf}{\mathbf}
\newcommand{\ds}{\displaystyle}

% From stackexchange
\DeclarePairedDelimiterX\set[1]\lbrace\rbrace{\def\given{\;\delimsize\vert\;}#1}
\DeclarePairedDelimiter\abs{\left \lvert}{\right \rvert}%

\DeclareMathOperator{\Ker}{Ker}
\DeclareMathOperator{\Null}{null}
\DeclareMathOperator{\range}{range}

\usepackage[parfill]{parskip}
\usepackage[margin=1in]{geometry}

\pagestyle{fancy}

\newcommand{\UU}{\mathcal{U}}
\newcommand{\T}{\text}

\newcommand{\eq}[1]{\begin{align*}#1\end{align*}}

\def\Ber{\text{Ber}}
\def\ub{\underbrace}
\def\UU{\mathcal{U}}
\def\WW{\mathcal{W}}
\def\XX{\mathcal{X}}
\def\VV{\mathcal{V}}
\def\Unif{\text{Unif}}
\def\Xh{\hat{X}}
\def\P{\text{P}}
\def\PP{\mathbb{P}}
\def\CC{\mathbb{C}}
\def\KK{\mathbb{K}}
\def\lb{\lambda}

\title{MATH 113 - Matrix Theory; Midterm Review}
\author{Instructor: Michael Kemeny; Notes: Adithya Ganesh}

\lhead{MATH 113}

\begin{document}

\maketitle
\tableofcontents


\section{Notes on 3.F: Duality}

\begin{definition}
  A linear functional on $V$ is a linear map from $V$ to $\mbf{F}$, i.e. an element of $\mathcal{L}(V, \mbf{F})$. \\
\end{definition}

\begin{definition}
  The dual space of $V$, denoted $V'$ is the vector space of all linear functions on $V$.  In other words, $V' = \mathcal{L}(V, \mbf{F})$.
\end{definition}

Note that $\dim V' = \dim V$.  This follows from 3.61, which states that  $\dim \mathcal{L}(V, W) = \dim (V) \dim(W)$.

\begin{definition}
  If $v_1, \dots, v_n$ is a basis of $V$, then the dual basis of $v_1, \dots, v_n$ is the list $\phi_1, \dots, \phi_n$ of elements of $V'$ where each $\phi_j$ is the linear functional on $V$ such that

  \[
    \phi_j(v_k) =
    \begin{cases}
      1; \qquad \text{ if } k = j \\
      0; \qquad \text{ if } k \neq j.
    \end{cases}
    \]
\end{definition}

%

\section{Review}

\begin{itemize}
  \item Let $T \in L(V, W)$.  Then $T$ is injective if and only if $\Null T = \left\{ 0 \right\}$.
  \item Fundamental theorem of linear maps.  Let $T$ be a linear map from $V$ to $W$.  Then
    \[
      \dim V = \dim \Null T + \dim \range T
      \]
    \item Matrix of a linear map.  Suppose $T \in L(V, W)$ and $v_1, \dots, v_n$ is a basis of $V$ and $w_1, \dots, w_m$ is a basis of $W$.  The matrix of $T$ w.r.t. these bases is the $m \times n$ matrix whose entries $A_{j, k}$ are defined by

      \[
        Tv_k = A_{1, k} w_1 + \dots + A_{m, k} w_m.
      \]

      (Look at picture on pg. 71).

    \item A linear functional on $V$ is a linear map from $V$ to $F$.

    \item The dual space of $V$, denoted $V'$, is the vector space of all linear functionals on $V$, with $V' = L(V, F)$.  Note that $\dim V' = \dim V$.

    \item If $v_1, \dots, v_n$ is a basis of $V$, then the dual basis of $v_1, \dots, v_n$ is the list $\phi_1, \dots, \phi_n$ of elements of $V'$, where each $\phi_j$ satisfies $\phi_j(v_k) = 1$ if $k = j$, else $0$.

    \item Suppose $V$ is finite dimensional and $U$ is s subspace of $V$.  Then
      \[
        \dim U + \dim U^0 = \dim V.
      \]

    \item (3.5) Suppose $v_1, \dots, v_n$ is a basis of $V$ and $w_1, \dots, w_n \in W$.  Then there exists a unique linear map $T: V \to W$ such that
      \[
        Tv_j = w_j
      \]
      for each $j = 1, \dots, n$.

    \item Volume functions.  A function $f: \RR^2 \times \RR^2 \to \RR$ is a volume function if it satisfies bilinearity and alternation.
      \begin{itemize}
        \item Bilinearity: $f(u, av + bw) = a f(u, v) + b f(u, w)$ and $f(au + bv, w) = a f(u, w) + b f(v, w)$.
        \item Alternation, $f(u, u) =0$.
      \end{itemize}

    \item Note that volume functions are a vector space.

    \item In general, can write volume functions in general for multilinear functions.

    \item Swapping two vectors changes the sign of the volume.

    \item For any permutation, define $inv(\sigma)$ to be the number of inversions.  The number of times that $i < j$, but $\sigma(i) > \sigma(j)$.

    \item For any permutation, define the sign as $(-1)^{inv(\sigma)}$.

    \item If $A_{ij}$ is an $n \times n$ over a field $k$, then $\det A$ is
      \[
        \det A = \sum_{\sigma \in \Sigma_n} \sign(\sigma) A_{\sigma(1), 1} \dots A_{\sigma(n), n}
      \]

    \item Q1a.  If $T$ is injective, then $Tv_1 = Tv_2$ implies $v_1 = v_2$.  To show linear independence, we claim that
      \[
        a_1 Tv_1 + a_2 Tv_2 + \dots + a_n T v_n = 0
        \]
        implies $v_i = 0$ for all $i$.

    \item Q2.

\end{itemize}

\end{document}
