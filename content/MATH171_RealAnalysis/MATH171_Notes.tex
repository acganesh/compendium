\documentclass[12pt]{article}

\usepackage{amsmath}
\usepackage{amssymb}
\usepackage{fancyhdr}
\usepackage{todonotes}
\usepackage{amsthm}
\usepackage{amsopn}
\usepackage{amsfonts}
\usepackage{mathtools}
\usepackage{libertine}

\newtheorem*{theorem}{Theorem}
\newtheorem*{lemma}{Lemma}
\newtheorem*{definition}{Definition}
\newtheorem*{remark}{Remark}
\newtheorem*{claim}{Claim}
\newtheorem*{example}{Example}
\newtheorem*{prop}{Proposition}
\newtheorem*{sol}{Solution}

\usepackage{latexsym}
\usepackage{bbm}
\usepackage[small,bf]{caption2}
\usepackage{graphics}
\usepackage{epsfig}
\usepackage{amsopn}
\usepackage{url}

\usepackage[parfill]{parskip}
\usepackage[margin=1in]{geometry}

\newcommand{\bc}{\binom}
\newcommand{\bx}{\boxed}
\newcommand{\RR}{\mathbb{R}}
\newcommand{\R}{\mathbb{R}}
\newcommand{\QQ}{\mathbb{Q}}
\newcommand{\II}{\mathbb{I}}
\newcommand{\Ra}{\mathcal{R}}
\newcommand{\EE}{\mathbb{E}}
\newcommand{\HH}{\mathcal{H}}
\newcommand{\NN}{\mathbb{N}}
\newcommand{\FF}{\mathbb{F}}
\newcommand{\ve}{\varepsilon}
\newcommand{\eps}{\epsilon}
\newcommand{\la}{\langle}
\newcommand{\ra}{\rangle}
\newcommand{\mbf}{\mathbf}
\newcommand{\ds}{\displaystyle}
\newcommand{\ol}{\overline}

% From stackexchange
\DeclarePairedDelimiterX\set[1]\lbrace\rbrace{\def\given{\;\delimsize\vert\;}#1}
\DeclarePairedDelimiter\abs{\left \lvert}{\right \rvert}%

\DeclareMathOperator{\sign}{sign}
\DeclareMathOperator{\Aut}{Aut}
\DeclareMathOperator{\GL}{GL}
\DeclareMathOperator{\Ker}{Ker}
\DeclareMathOperator{\im}{im}
\DeclareMathOperator{\Syl}{Syl}
%\newcommand{\mat}[4]{\begin{pmatrix} #1 & #2 \\ #3 & #4\end{pmatrix}}
\newcommand*{\mat}[1]{\begin{pmatrix}#1\end{pmatrix}}


\usepackage[parfill]{parskips
\usepackage[margin=1in]{geometry}

\pagestyle{fancy}

\newcommand{\UU}{\mathcal{U}}
\newcommand{\T}{\text}

\newcommand{\eq}[1]{\begin{align*}#1\end{align*}}

\def\Ber{\text{Ber}}
\def\ub{\underbrace}
\def\UU{\mathcal{U}}
\def\WW{\mathcal{W}}
\def\XX{\mathcal{X}}
\def\VV{\mathcal{V}}
\def\Unif{\text{Unif}}
\def\Xh{\hat{X}}
\def\P{\text{P}}
\def\PP{\mathbb{P}}
\def\CC{\mathbb{C}}
\def\KK{\mathbb{K}}
\def\ZZ{\mathbb{Z}}
\def\lb{\lambda}
\def\rot{\text{rot}}

\title{MATH 171 - Real Analysis}
\author{Instructor: George Schaeffer; Notes: Adithya Ganesh}

\lhead{MATH 171}

\newcommand*{\mat}[1]{\begin{pmatrix}#1\end{pmatrix}}

\begin{document}
\maketitle

\tableofcontents

\section{9-24-18: Everything is a set}

Administrivia:
\begin{itemize}
    \item Book: Johnsonbaugh and Pfaffenberger
    \item (Supplement) Rudin's Principles of Mathematical Analysis
    \item Exam: likely week 5.
\end{itemize}

\subsection{On sets}
One motivation for analysis is a problem identified in 1901: Russell's Paradox.  Consider
\[
  R = \left\{ x : x \not \in X \right\} = \text{ the set of all sets that do not contain themselves }
\]

Problem: does the set contain itself?  Either $R \in R$ or $R \not \in R$, but neither is possible.

Rules for what is isn't a set: Zermelo-Frankel axioms.

In particular, under ZF: Can't build $\{x : x \text{ has property } P \}$.  You must say
\[
  \left\{ x \in S: x \text{ has property } P \text{ where } S \text{ is already a set. } \right\}
\]

But going further: the collection of all sets is itself not a set.

Axioms of choice (the Cartesian product of a collection of non-empty sets is non-empty).

We can define the natural numbers in the framework of sets.  If $x$ is a set we can define its successor as $S(x) = x \cup \left\{ x \right\}$.

\begin{itemize}
  \item 0 = $\emptyset$
  \item 1 = $\left\{ \emptyset \right\}$
  \item 2 = $\{ \left\{ \emptyset \right\}, \emptyset \}$.
  \item 3 = $\{ \left\{ \emptyset \right\}, \emptyset, \left\{ \left\{ \left\{ \emptyset \right\}, \emptyset \right\} \right\}. = \left\{ 0, 1, 2 \right\}$.
\end{itemize}

\subsection{On functions (and cartesian products)}

{\it Cartesian Product.} Let $X$ and $Y$ be sets.  Then we can write
\[
  X \times Y = \left\{ (x, y) : x \in X, y \in Y \right\}.
\]

How do we define ordered pairs?  $(x, y) \neq \left\{ x, y \right\} = \left\{ y, x \right\}$ doesn't work, since order matters.

Instead, we want to say

\[
  (x, y) = \left\{ x, \left\{ x, y \right\} \right\}.
\]

What is a function?  We can write $f: X \to Y$, where $X$ is the domain(f) and $Y$ is the codomain(f).  A function $f: X \to Y$ is a subset of $X \times Y$ satisfying the following:

\begin{itemize}
  \item $\forall x \in X, \exists y \in Y: (x, y) \in f$.
  \item  $\forall x \in X, \forall y, y' \in Y: (x, y) \wedge (x, y') \in f \implies y = y'$.
\end{itemize}

As a set, for example, $\sin \subseteq \R \times \R$.

\subsection{On natural numbers}

The set $\NN$ is equipped with a sucessor function $S: \NN \to \NN: x \mapsto S(x)$.  There are a few rules attached to this, namely the Peano axioms:

\begin{itemize}
  \item $\forall x \in \NN: S(x) \neq 0$
  \item $S$ is ``injective'': If $S(x) = S(y) \implies x = y$.
  \item Axiom of induction: If $K \subseteq \NN$ satisfying
    \begin{itemize}
        \item $0 \in K$
        \item $\forall x \in K, S(x) \in K$.
    \end{itemize}
\end{itemize}

$\NN$ has two binary operations, $+, \cdot$, addition and multiplication.

A binary operation on $X$ is a function $X \times X \to X$.

\begin{itemize}
  \item + (a, b) = a+b
  \item \cdot (a, b) = ab
  \end{itemize}


  $\forall a, a + 0 = a$. $\forall a, b; a + S(b) = S(a+b)$.


  \section{10-1-18: Suprema and infima}


  {\it Theorems of \RR.} 

  \begin{itemize}
    \item $\RR$ is an ordered field.
    \item Tere are lots of ordered fields: $\QQ$.
    \item Least upper bound axiom: If $S \subseteq \RR$ is nonempty and bounded above, then $S$ has a least bound $\in \RR$.
  \end{itemize}

  \begin{definition}
    Let $S \subseteq \RR, M \in \RR$.  We say that M is an upper bound on $S$ is $\forall x \in S: x \leq M$.  $M$ is the least upper bound (or the supremeum) of $S$ is $\forall M' < M$, $M'$ is not an upper bound on $S$.
  \end{definition}

  Furthermore: $M = \sup(S)$ if

  \begin{itemize}
    \item $M$ is an upper bound on $S$:
    \item $\forall M' < M:$ $M'$ is not an upper bound on $S$.
      \begin{align*}
        & \neg [ \forall x \in S : x \leq M'] \\
        & \exists x \in S: \neg [x \leq M' ] \\
        & \exists x : S: x > M' \\
        & \boxed{\forall \epsilon > 0, \exists x \in S: x > M - \epsilon}
      \end{align*}
  \end{itemize}

  Easy two step process for proving $M = \sup(S)$.

  The ``greatest lower bound'' axiom is equivalent to the ``least upper bound'' axiom.  Note that for convenience $\sup (\text{unbounded above } S) = + \infty$ and $\sup(\emptyset) = - \infty$.

  Consequences of the axioms in $\RR$.

     {\it Archimedean Property.} $\forall a, b \in \RR, a, b > 0$, then $\exists n \in \NN$ such that $na > b$.

     \begin{proof}
       Let $S = \left\{ n \in \NN : na \leq b \right\}$; which implies $n \leq \frac{b}{a}$.  $S$ is nonempty because $0 \in S$.  $S$ is bounded above by $\frac{b}{a}$.  By LUBA: $S$ has a supremum $m = \sup(S)$.  $m + 1 \not \in S$ and $m+1 \in \NN$ (left as an easy exercise).  
       
       Why is $m+1 \not \in S$?  Otherwise $m+1 \leq m$.  So $\neg [(m+1)a \leq b]$, i.e. $(m+1) a > b$.
     \end{proof}

     Note that the Archimedean principle is true in $\QQ$ as well.  It inherits AP from R.  There is also an independent proof just using the construction of $\QQ$ as fractions.\footnote{On HW2: An example of an ordered field, for which the AP fails.}

     \begin{theorem}
       The rational numbers form a dense subset of $\RR$.
     \end{theorem}

     We start by explaining the definition: $\forall a, b \in \RR: a < b \implies [ \exists r \in \QQ: a < r < b]$.  We now mention a lemma that will help us prove the theorem. 

     \begin{lemma}
       If $a < b \in \RR$ and $b - a > 1$, then $\exists n \in \ZZ: a < n < b$.
     \end{lemma}

     \begin{proof}
       Let $S = \left\{ n \in \ZZ : n \leq a \right\}$.

     By LUBA, we let $m = \sup(S)$.  Note that $m \in \ZZ$.  $m+1 \not \in S$.  We can easily verify that $a < m+1 < b$.  The first inequality follows from $m+1 \not \in S$, and for the second, note that:

     \begin{align*}
       m+1 &< m+(b-a) \\
       & \leq a + (b-a) = b.
     \end{align*}


    Here, we have used the fact that $m \in S$, so $m \leq a$.
     \end{proof}

     \begin{proof}(Main Theorem.)
       We know $a < b$.  By the Archimedean Principle, since $b-a > 0$ and $1 > 0$, there must exist $n \in \NN$ such that $n(b-a) > 1$.  This implies that $nb - na > 1$.  Also $na < nb$.

       By the lemma, there exists an integer $k \in \NN$ with $na < k < nb$.  Dividing by $n$, we obtain the fraction $\frac{k}{n}$ which satisfies $a < \frac{k}{n} < b$.
     \end{proof}

     The irrationals are also dense in the reals - just take the rationals and add $\sqrt{2}$, and follow a similar argument.


     \section{Continuity - 10-19}

     If $(X, \tau)$ is a topological space and $S \subseteq X$, we can give $S$ a topology
     \begin{align*}
       \tau_S = \left\{ U \cap S : \text{ where } U \in \tau_X \right \}. 
     \end{align*}

     This is a subspace / induced / inherited / relative topology on $S$.

     Note: $[0, 1]$ w/ the subspace topology from $\RR$.


     If $(X, d_X)$ is  ametric space, $S \subseteq X$, then $S$ is also a metric space, where $d_S = d_X | _{S \times S}$ (restricted for $S \times S)$.


     If $X$ is a metric space and $S \subseteq X$, then the topoloy from $d_$ is the subspace topology.


     If $U$ is open / closed in $S$, it need not be closed in $X$.


     However, if $K$ is compact in $S$, then $K$ is compact in $X$.

     {\it Self explanatory (?)}  We say that a topological space $X$ is compact if it is a compact set in its own topology.

     {\it Continuous functions.} Let $(X, d_X)$ and $(Y, d_Y)$ be metric spaces, let $f: X \to Y$, let $p \in X$.  We say that $f$ is continuous at $p$
     \begin{enumerate}
       \item In the analytic sense if 
         \begin{align*}
           \forall \eps > 0, \exists \dl > 0, \forall x \in X: d_{X}(x, p) < \dl \implies d_Y (f(x), f(p)) < \eps.
         \end{align*}
       \item In the sequential sense if $\forall$ sequences $(x_n)_n \to p \in X$, the sequence
         \begin{align*}
           (f(x_n))_n \to f(p).
         \end{align*}
       \item In the topological sense if $\forall $ open $V \subseteq Y$, if $f(p) \in V$, then there exists an open $U \subseteq X$ such that $p \in U$ and $f(U) \subseteq V$.
     \end{enumerate}

     We will show that all of these are equivalent.

     {\bf Notes on these definitions.}

     \begin{itemize}
       \item For the topological sense: can switch ``open'' for ``closed.''  
       \item Also, (iii) is the definition of ``continuous' at p'' for general topological spaces.  It doesn't require a metric on $X$ or $Y$.
        \item In (i), $\delta$ can depend on both $p$ an $\eps$.
     \end{itemize}

    {\bf Outline of proof of equivalence.}

    \begin{itemize}
      \item $1 \implies 2$ (easy; definition pushing).
      \item $2 \implies 1$ is slightly harder.  We will prove this by contrapositive.  We'll show: if $f$ is not analytically continuous at $p$, then it's not sequentially cts.
        \begin{align*}
          \forall \varepsilon > 0, \forall \delta > 0; \exists x \in X: d_X(x, p) < p \text{ and } d_Y (f(x), f(p)) \geq \varepsilon.
        \end{align*}

        In particular, we can define $(x_n)_{n=1}^{\infty}$ so that
        \begin{align*}
          d_X(x_n, p) < \frac{1}{n} \text{ and } d_Y (f(x_n), f(p)) \geq \varepsilon.
        \end{align*}

        This means that $x_n \to p$, but $f(x_n) \not \to f(p)$.
      \item $3 \implies 1$. Let $\varepsilon > 0$. We know $f$ is topologically continuous, so let $V = B_{\varepsilon}(f(p))$ is open.  Then there exists an open $U \ni p$ such that $f(U) \subseteq V$.  Because $U$ is open, $p$ is interior to $U$, so $\exists \delta$ such that $B_{\delta}(p) \subseteq U$.

        So, $x \in B_{\delta}(p) \implies x \in U \implies f(x) \in V \implies f(x) \in B_{\varepsilon} (f(p))$.

      \item $1 \implies 3$ is similar to the previous, just reverse all the statements.
    \end{itemize}

    \begin{definition}
      The function $f: X \to Y$ is continuous if it is continuous at all $p \in X$.
    \end{definition}

    Translated definitions to ``everywhere continuous.''
    \begin{itemize}
      \item Analytic continuity is easy.
      \item Sequential continuity: $\forall$ convergent sequences $(x_n)_n, (f(x_n))_n$ is convergent and $\lim f(x_n) = f(\lim x_n)$.
      \item Topological continuity: $f$ is continuous if for all open $V \subseteq Y$, $f^{-1}(V)$ is open in $U$.
    \end{itemize}

For example, consider $f(x) = x^2$.  Note that $f\left( -1, 1))  \right) = [0, 1)$.  If $f$ is continuous and $U$ is open $f(U)$ may not be open.

Also, look at $g(x) = \frac{1}{x}$ on $(0, \infty)$.  Consider $C = \mathbb{Z}_{> 0}$, and then look at the set $g(C) = \left\{ \frac{1}{n} : n \geq 1 \right\}$ is not closed.

Continuous functions don't preserve openness and they don't preserve closedness, but they preserve compactness.

\begin{theorem}
  If $f: X \to Y$ is continuous, (for $(X, Y)$ topological spaces) and $K \subseteq X$ is compact in $X$, then $f(K)$ is compact in $Y$.
\end{theorem}

\begin{proof}
  Need to show that every open cover of $A$ has a finite subcover.  Let $H$ be an open cover of $f(K)$.  Let $G = \left\{ f^{-1}(V) : V \in H  \right\}$.  Now, $G$ covers $K$.  Since $f$ is continuous, it is an open cover\footnote{Recall that a topological space $X$ is called compact if each of its open covers has a finite subcover.  That is, $X$ is compact if for every collection $C$ of open subsets of $X$ such that 
 \[
   X = \bigcup_{x \in C} x,
   \]
  there is a finite subset $F$ of $C$ such that
  \[
    X = \bigcup_{x \in F} x.
    \]
}
\end{proof}

{\it Corollary.} (Extreme value theorem).  If $f: X \to \RR$ ($X$ is a compact topological space, $f$ is continuous).  Then $f$ achieves a maximum and minimum on $X$.  Meaning, there exists $p, q \in X$ such that $\forall x \in X, f(p) \leq f(x) \leq f(q)$.

\begin{proof}
  $f(X)$ is a compact $\subseteq \RR$, so $f(X)$ is closed and bounded (Heine-Borel). Provided $X \neq \emptyset, f(X) \neq \emptyset$. Now, 
  \begin{itemize}
    \item $m = \inf (f(X) ) \in f(X)$ and $M = \sup (f(X) ) \in f(X)$
    \item By definition, since $m, M \in f(X)$, $\exists p, q \in X: f(p) = m$ and $f(q) = M$.
  \end{itemize}.
\end{proof}

The topological proof is surprisingly fast.  Indeed, you can prove this using the sequential definition of continuity using the Bolzano-Weierstrass; but it is tricky.


Some remarmks on Heine-Borel:\footnote{$f(X)$ is compact $\subseteq \RR$, so $f(X)$ is closed and bounded (true in any metric space).  The other direction requires being a subset of $\RR^n$, which requires Heine-Borel}.

Next week: we'll discuss the notion of ``connectedness.''

Take home exam: goes out after class on Wed, have until Friday to finish.

\section{10-22: Connectedness}

Exam: released on Wednesday after class.  You'll have 3 hours + extra time to submit.  Can start at any time until midnight - $\eps$.  Open textbooks (J\&P, Rudin), + notes.

\begin{definition}
  Let $X$ be a topological space.  $X$ is called disconneted if there are nonempty, disjoint open sets $U$ and $U'$ of $X$ such that $X = U \cup U'$.  If $X$ is not disconnected, it's called connected. \\
\end{definition}

\begin{definition}
  If $X$ is a topological space and $S \subseteq X$, then $S$ is ``connected'' if $S$ is connected as a topological space (with respect to the subspace topology). \\
\end{definition}

\begin{definition}[Alternative]
  A topological space is connected if the only clopen sets are $X$ and $\emptyset$. \\
\end{definition}

We now ask: what are the connected subsets of $\RR$? \\

\begin{lemma}
  $S \subseteq \RR$ is connected iff $S$ is an interval.  $\forall x, y \in S, \forall z \in \RR, x < z < y \implies z \in S$. \\
\end{lemma}

\begin{proof}
  We start with the forward direction.  Assume $S$ is not an interval.  Then there exists some $z \in \RR$ so that $x < z < y$ but $z \not \in S$.  Let $U = (- \infty, z)$ and let $U' = (z, \infty)$.  Then it is easy to check that $(S \cap U) \cup (S \cap U')$ is a disconnection of $S$.

  Need to check:
  \begin{itemize}
    \item $S \cap U$ is open in $S$, because $U$ and $U'$ are open in $\RR$.
    \item The sets $S \cap U$, $S \cap U'$ are disjointed, because $U, U'$ are disjoint.
    \item $S = (S \cap U) \cup (S \cap U')$.
  \end{itemize}

  If $(X, \tau)$ is a topological space and $S \subseteq X$, the subspace topology on $S$ is 
  \begin{align*}
    \tau_{S} = \left\{ S \cap U: U \in \tau \right\}.
  \end{align*}

  Also, $\tau_S$ is the coarsest topology such that $S \to X$ inclusion is continuous.

  Now, we move on to the reverse direction.  Let $S$ be an interval, so that
  \begin{align*}
    S = (S \cap U) \cup (S \cap U'),
  \end{align*}
  where $U$ and $U'$ are open in $\RR$, $S \cap U$ and $S \cap U'$ are nonempty.  Want to show that they are not disjoint.  Let $V = S \cap U, V' = S \cap U'$.

  Let $x \in V$ and $y \in V'$.  Without loss of generality, assume $x < y$.  Also, $\frac{x+y}{2} \in S$, since $S$ is an interval.

  We will construct sequences $(x_n)_n$ and $(y_n)_n$ as follows.

  \begin{itemize}
    \item $x_0 = x$ and $y_0 = y$.
    \item Let $\alpha_{n+1} = \frac{x_n + y_n}{2}.$  If $\alpha_{n+1} \in V$, then $x_{n+1} = \alpha_{n+1}$ and $y_{n+1} = y_n$.  Otherwise $\alpha_{n+1} \in V'$ and $x_{n+1} = x_n$ and $y_{n+1} = \alpha_{n+1}$.
    \item $(x_n)_n$ is an increasing sequence in $V$, and $(y_n)_n$ is a decreasing sequence in $V$.  They are also bounded, since they are termwise bounded by each other.
    \item $(x_n)_n$ and $(y_n)_n$ both convergence, and since
      \begin{align*}
        |x_n - y_n| \leq 2^{-n} |x - y|,
      \end{align*}
      both sequences convergence to the same limit $L$; $x < L < y \implies L \in S$.  Therefore, $L \in V$ or $L \in V'$.

    \item Suppose $L \in V$.  Then $L \in U$.  $L$ is an interior to $U$ (since $U$ is open).  In particular, $\exists \eps > 0$ such that $B_{\eps}(L) \subseteq U$.  By the convergence of $(y_n)_n \to L$, $\exists N$ such that
      \begin{align*}
        |y_n - L| < \eps; \forall n \geq N
      \end{align*}

      In particular, $y_N \in B_{\eps}(L) \subseteq U \implies y_n \in V$.  So since $y_N \in V'$, $V \cap V' \neq \emptyset$.
  \end{itemize}
\end{proof}

\begin{example}
  A disconnected set in $\RR$: $\mathbb{Q}$ is disconnected.
\end{example}

Consider a dramatic example: the Cantor set.

\begin{itemize}
  \item In homework, proved that every open ball is closed in an ultrametric space.
  \item The Cantor set is ``totally disconnected\footnote{a lot of open sets are closed}''  Turns out that every ultrametric space is topologically equivalent to the Cantor set.
\end{itemize}

Last time: Let $f: X \to Y$ be a continuous function of topological spaces.  If $X$ is compact, then $f(X)$ is compact.  This implies the Extreme Value Theorem.

This time: Let $f: X \to Y$ be a continuous function of topological spaces.  If $X$ is connected, then $f(X)$ is connected.

\begin{proof}
  Suppose $f(X)$ is disconnected.  then
  \begin{align*}
    f(X) = (f(X) \cap V) \cup (f(X) \cap V').
  \end{align*}
  Let $W = f(X) \cap V$, $W' = f(X) \cap V'$.  Let $U = f^{-1}(W)$ and $U' = f^{-1}(W')$.

  \begin{itemize}
    \item $U \cap U' = X$ (by definition of preimage).
    \item $U$ and $U'$ are open, because $f^{-1}(W) = f^{-1}(f(X) \cap V) = f^{-1}(V)$.  Further, $f^{-1}(V)$ is open because $f$ is continuous and $V$ is open.
    \item They're nonempty because $W, W'$ are nonempty $\subseteq f(X)$.
    \item They're disjoint because if $x \in U \cap U'$, then $f(x) \in W \cap W'$; but $W$ and $W'$ are disjoint.
  \end{itemize}
\end{proof}

{\it Corollary.} (Intermediate value theorem.)  Let $X$ be a connected topological space, and let $f: X \to \RR$ be a continuous real-valed function.  If there are $p, q \in X$ and $c \in \RR$ such that $f(p) < c  < f(q)$, $\exists \xi \in X$ such that $f(\xi) = c$.

\begin{proof}
  $f(X)$ is an interval.
\end{proof}

\begin{example}[Incomplete topologist's sine curve] 
  Consider the graph of $\sin \left( \frac{1}{x} \right)$ for $x \in \left( 0, 1 \right)$.  Note that $\sin \left( \frac{1}{x} \right)$ is continuous on this interval.  This is connected.
\end{example}

\begin{example}[Midcomplete TSC] 
  Consider $\left\{ \text{Incomplete TSC} \right\} \cup \left\{ (0, 0) \right\}$.  This will be connected, still.  But, it is not path connected.  Consider a point $(x, y)$; there is no path between $(x, y)$ and $(0, 0)$.

  Interestingly, this is a converse to the Intermediate Value Theorem.  Has the intermediate value property, but it is not continuous at $0$.
\end{example}


\section{10-24: Uniform continuity}

Exam will be ready at 12:30pm.  Have 3 hours + 30 extra minutes to scan + upload.

Today: just one proof, and then Q\&A time / review.

\begin{theorem}
  Let $f: X \to Y$ be a continuous function with $X$ and $Y$ metric spaces.  If $X$ is compact then $f$ is uniformly continuous.
\end{theorem}

\begin{enumerate}
  \item Recall that $f:X \to Y$ is continuous if $\forall p \in X, [\forall \varepsilon > 0, \exists \delta > 0, \forall x \in X$
  \begin{align*}
    d_X(x, p) < \delta \implies d_Y (f(x), f(p)) < \varepsilon.
  \end{align*}
\item $\forall p \in X$, $\forall (x_n)_n \to p$ if $f(x_n)_n$ converges $\to f(p)$.
\item $\forall$ open $U \subseteq V f^{-1}(V)$ is open in $X$. ``preimage of an open set is open.''
\end{enumerate}

Importantly, 1, 2, 3, work in metric spaces, and 3 works in topological space.

{\bf Continuity.}

In general, when we talk about continuity, we are discussing conditions of the form $\forall p \in X, \forall \varepsilon > 0, \exists \dl > 0$ such that $\forall x \in X [\dots]$.

We say that $f: X \to Y$ (metric spaces) is uniformly continuous if
\begin{align*}
  \forall \varepsilon > 0, \exists \dl > 0, \forall x, p \in X: d_X(x, p) < \delta \implies d_{Y}(f(x), f(p)) < \eps.
\end{align*}

The salient difference here is that $\delta$ only depends on $\varepsilon$, and no longer depends on $p$.

{\bf Obvious implication.} If $f$ is uniformly continuous, it is continuous.  The converse, however is false.

{\bf Example.} Let $f(x) = \frac{1}{x}$ on $(0, + \infty)$. This is a continuous function (on the interval).  It is not uniformly continuous.

Consider some point $(p, f(p))$.  Suppose we have a range $(f(p) - \eps, f(p) + \eps)$.  Need a delta such that whenever $x \in (p - \delta, p + \delta)$, $f(x) \in (f(p) - \eps, f(p) + \eps)$.  As $p \to 0$, $\delta$ stops working (since it will contain the asymptote at $0$).

{\bf Example.} Let $f(x) = \sin \left( \frac{1}{x} \right)$ on $(0, + \infty)$.  This function is continuous, but not uniformly so.  No matter how small we make $\delta$, there is some point $p$ close to 0 so that $f((p - \delta, p + \delta)) = [-1, 1]$.

Aside: the difference between Lipschitz continuity and uniform continuity. \footnote{You can show that $\sqrt{x}$ is uniformly continuous, but not Lipschitz continuous.}

{\bf Proof of theorem.} Suppose that $f: X \to Y$ is continuous but not uniformly continuous (where $X$ is compact).  Then $\exists \varepsilon > 0$, $\forall \dl > 0$, $\exists x, p \in X$ such that $d_{X}(x, p) < \dl$ and $d_{Y} (f(x), f(p)) \geq \eps$.  We want to use the above to build two sequences $(x_n)_n$ and $(p_n)_n$ such that $\forall n \geq 1$,

\begin{align*}
  d_X(x_n, p_n) < \frac{1}{n}; \qquad d_Y(f(x_n), f(p_n)) \geq \varepsilon
\end{align*}

We have not used compactness yet.  By sequential compactness: some subsequence of $(x_n)$ converges.  Some subsequence of $(x_n)_n$ converges to $L \in X$, so that $(x_{n_i})_i \to L$.  Now, consider $(p_{n_i})$; we also have $(p_{n_i})_i \to L$.

Therefore
\begin{align*}
  \lim_{i \to \infty} f(x_{n_i}) = f(\lim (x_{n_i})_i) = f(L) = f(\lim_{i \to \infty} (p_{n_i})_i) = \lim_{i \to \infty} f(p_{n_i}).
\end{align*}
(for the first equality we have used continuity).  But this is a contradiction, since our sequences are actually far apart.

Recall the theorem that states that if $f: [a, b] \to \RR$ is continuous, then it is integrable.  The proof of this theorem relies on the notion of uniform continuity.

\subsection{Review}

We now discuss the sequential version of uniform continuity. \\

\begin{theorem}
  Let $f: X \to Y$ be continuous metric spaces.  Then the following are equivalent:

  \begin{itemize}
    \item Let $f$ is uniformly continuous.
    \item If $(x_n)_n$ is a Cauchy sequence, then $f(x_n)_n$ is also Cauchy.
  \end{itemize}
\end{theorem}

\begin{theorem}
  If $X$ is a metric space and $K \subseteq X$ is compact and $(x_n)_n$ is a sequence in $K$, it has a convergent subsequence whose limit is in $K$.
\end{theorem}

{\bf Dense sets.} Suppose $X$ is a metric space $S \subseteq X$, $S$ is dense if 

\begin{itemize}
  \item $\ol{S} = X$
  \item Every nonempty open $U$ overlaps with $S$. 
  \item For all $x \in X$ and $\forall \eps > 0$, $\exists y \in S$ such that $d_{X}(x, y) < \eps$.
\end{itemize}

For example, $\QQ$ is dense in $\RR$ (that is, all real numbers can be arbitrarily well approximated by rational numbers).

If $X$ is a metric space with a countable dense set, then we call $X$ separable.  This is good because this means that computers can deal with such sets quite well (e.g. floating point).

{\bf Theorem.} In the space of continuous functions $[0, 1] \to \RR$, the rational coefficient polynomials are dense.


\section{11-05-18: Integrability and FTCs}

Recall the Riemann-Darboux integral.  Suppose $f:[a, b] \to \R$ is bounded with $a < b$.  Then we can write

\[
  \int_{\ol{a}}^{b} = \sup \left\{  \int_{a}^{b} \varphi: \varphi \text{ a step fn }; \varphi \leq f \right\}
\]
or
\[
  \int_{a}^{\ol{b}} = \inf \left\{ \int_{a}^{b} \psi: \psi \text{ a step fn and } \psi \geq f \text{ on } [a, b] \right\}.
\]

And if they match, $f$ is integrable and $\int_{f} = \int_{\ol{a}}^{b} f = \int_{a}^{\ol{b}} f$.

\begin{prop}[Sequential criterion for integrability] Suppose $f: [a, b] \to \R$ is bounded, $L \in \R$.  
  
  Subtext: Wts 
  \[
    \int_{a}^{b} = L.
  \]

  Then $\int_{a}^{b} f = L$ if and only if $\exists (\psi_n)_n, (\spi_n)_n \in \text{Step}([a, b])$ for all $k$,  $\psi_k \leq f \leq \varphi_k$; and
  \[
    \int_{a}^{b} \psi_n \to L; \qquad \int_{a}^{b} \varphi_n \to L.
  \]
\end{prop}

{\bf Sufficient conditions for integrability.}
  \begin{itemize}
    \item If $f: [a, b] \to \R$ is continuous, it is integrable.  
    \item Piecewise continuous.
    \item Monotonic functions.
    \item ``Piecewise monotone and/or continuous\dots''
    \item Thomae's function shows that the converse is not true.
  \end{itemize}

  \begin{prop}[Cauchy criterion for integrability.]
    If $f: [a, b] \to \R$ is bounded, it is integrable iff $\forall \varepsilon > 0$, there exists $\psi, \varphi \in \text{Step}([a, b])$ such that $\varphi \leq f \psi$ and $\int_{a}^{b} (\varphi - \psi) = \int_{a}^{b} \varphi - \int_{a}^{b} \psi < \varepsilon$.

    (Proof follows from the definition of integrability.)
  \end{prop}

  \begin{theorem}
    If $f: [a, b] \to \R$ is continuous, then it is integrable.
  \end{theorem}

  \begin{proof}
    Strategy: for any interval $I$ we want $\varphi(I) - \psi(I)$ to be small.

    Let $\varepsilon > 0$.  Pick $\delta > 0$ such that $\forall x, y \in [a, b]$, we have
    \begin{align*}
      |x - y| < \delta \implies |f(x) - f(y)| < \frac{\varepsilon}{b-a}.
    \end{align*}

    Pick a partition of $[a, b]$ into disjoint intervals $\left\{ I_k \right\}_{k=1}^{n}$ with $|I_k| < \delta$.  For each $I_k$, $f: \ol{I_k} \to \R$ achieves its min and max at $p_k, q_k \in \ol{I}_k$ respectively with
    \begin{align*}
      f(p_k) \leq f(x) \leq f(q_k)
    \end{align*}
    for all $x \in I_k$.

    Now, let
    \[
      \varphi = \sum_{k=1}^{n} f(p_k) \mathbf{1}_{I_k}; \qquad \psi = \sum_{k=1}^{n} f(q_k) \mathbf{1}_{I_k}.
    \]

    \todo{Fill in details from notes.}


  \end{proof}

  \begin{theorem}[Lebesgue's Riemann integrability condition]
    A function $f$ is integrable if and only if
    \[
      \lambda \left( \left\{ p \in [a, b]: f \text{ is discontinuous at p } \right\} \right) = 0,
      \]
      that is the set of discontinuities has Lebesgue measure 0 (alternatively, $f$ is almost everywhere continuous).
  \end{theorem}

  \footnote{Cannot use this on homework unless you prove it.} This theorem implies:
  \begin{itemize}
    \item $f$ is continuous implies it is integrable.
    \item Monotone functions are continuous.
  \end{itemize}

  Let $f: [a, b] \to \R$ be bounded, we call $F:[a, b] \to \R$ a [continuous] antiderivative of $f$ if it is continuous on $[a, b]$, differentiable on $(a, b)$, and $\forall p \in (a, b): F'(p) = f(p)$.

  \begin{theorem}[The fundamental theorem of calculus]
    Let $f:[a, b] \to \R$ be an integrable function $F:[a, b] \to \R : x \mapsto \int_{a}^{x} f$.  Then:
    \begin{enumerate}
      \item $F$ is Lipschitz continuous.
      \item If $f$ is continuous at $p \in (a, b)$, then $F$ is differentiable at $p$ and $F'(p) = f(p)$.
      \item If $f$ is continuous on all of $[a, b]$, then $F$ is an antiderivative of $f$.
    \end{enumerate}
  \end{theorem}

  \begin{proof}
    The first statement is proven by the corresponding theorem for the upper / lower integrals.  Statement (3) follows from statement (2).

    Hence, it suffices to prove (2).  

    If $f$ is continuous at $p$, then
    \begin{align*}
      \lim_{x \to p^{+}} \frac{F(x) - F(p)}{x-p} = f(p).
    \end{align*}

    In these notes we will prove half of it (since the case $x \to p^{-}$ is similar.)  Make the change of variables $x = p+h$.  Then the limit above is equivalent to
    \begin{align*}
      \lim_{h \to 0^{+}} \frac{F(p+h) - F(p)}{h} &= \lim_{h \to 0^{+}} \left[ \frac{1}{h} \int_{p}^{p+h} f \right],
      \intertext{where}
      F(x) = \int_{a}^{x} f.
    \end{align*}

    Let $\delta > 0$ such that $\forall x \in [a, b]: \left|x-p\right| < \delta \implies \left|f(x) - f(p)\right| < \varepsilon$.  If $|x - p| < \delta$ then
    \begin{align*}
      f(p) - \varepsilon < f(x) < f(p) + \varepsilon.
    \end{align*}

    So, for $h < \delta$, we have
    \begin{align*}
      \int_{p}^{p+h} \left[ f(p) - \varepsilon \right] & \leq \int_{p}^{p+h} f(x) \\
      & \leq \int_{p}^{p+h}  \left[ f(p) + \varepsilon \right].
    \end{align*}

    This implies that for $h < \delta$,
     \begin{align*}
      h(f(p) - \varepsilon) \leq \int_{p}^{p+h} f \leq h (f(p) + \varepsilon);
    \end{align*}
    that is
    \begin{align*}
      f(p) - \varepsilon \leq \frac{1}{h} \int_{p}^{p+h} f \leq f(p) + \varepsilon.
    \end{align*}

    The punch line is that

    \begin{align*}
      \left|\lim_{h \to 0^{+}} \left[ \frac{1}{h} \int_{p}^{p+h} f \right] - f(p)\right| < \varepsilon.
    \end{align*}

    In particular, the limit above is equal to $f(p)$.  This proves the FTC for the derivative of the integral; Wednesday we will do the integral of the derivative.
  \end{proof}


  \section{Sequences and series of functions}

  \subsection{Pointwise vs. uniform convergence}

\newpage

\section{Key ideas}

{\bf Definition of a metric space.}

A metric space is a set $X$ together with a distance $d: X \times X \to \RR_{\geq 0}$ that satisfies

\begin{itemize}
  \item $d(x, x) = 0$.
  \item $d(x, y) = d(y,x)$.
  \item $d(x, y) + d(y, z) = d(x, z)$.
\end{itemize}

These three results together imply $d(x, y) \geq 0$.


{\bf Theorem: Cauchy-Schwarz.}

Let $a_1, \cdots, a_n, b_1, \dots, b_n \in \RR$.  Then
\begin{align*}
  \left[ \sum_{k=1}^{n} a_k b_k \right]^2 \leq \left[ \sum_{k=1}^{n}a_k^2 \right] \left[ \sum_{k=1}^{n} b_k^2 \right].
\end{align*}

To recall inequality, just recall that $||u|| ||v|| \cos \theta = u \cdot v$.

{\bf Convergence in a metric space.} 

A sequence $(x_n)_n$ is convergent to $L$ if $\forall \varepsilon > 0$, $\exists N$ such that $n > N \implies |x_n - L| < \varepsilon$.

{\bf Cauchy-ness in a metric space.}

$\forall \varepsilon > 0, \exists N$ s.t. $\forall m, n \geq N$, $d(x_n, x_m) < \varepsilon$.

{\bf Complete metric space, and an example.}

A metric space is called complete if Cauchy $\implies $ convergent.  $\RR$ is a complete metric space.

{\bf Convergence in $\RR^n$}.  

$\ol{a}^{k} \to \ol{a}$ iff $\ol{a}_j^{k} \to a_j$ for all $j$.

Similar proof for Cauchy in $\RR^n$.

{\bf Topological space.}  

A topological space is a $(X, \tau)$ where $X$ is a set and $\tau \subseteq P(x)$ satisfies

\begin{itemize}
  \item $\emptyset, X \in \tau$
  \item If $\mathcal{F} \subseteq \tau$, then $\bigcup_{S \in \mathcal{F}} S \in \tau$ (an arbitrary union of sets in $\tau$ is in $\tau$).
  \item If $U_1, U_2 \in \tau$, then $U_1 \cap U_2 \in \tau$ (the intersection of any sets in $\tau$ are in $\tau$).
\end{itemize}

Intuition \url{https://math.stackexchange.com/a/523794.}  Broadly: a topology defines a notion of nearness on a set.

{\bf Interior / adherent sets.}

Let $(X, d)$ be a metric space, $x \in X, S \in X$.  Then $X$ is interior to $S$ if $\exists \varepsilon > 0, B_{\varepsilon} \subseteq S$.

$X$ is adherent to $S$ if $\forall \varepsilon > 0, B_{\varepsilon}(x) \cap S \neq \emptyset$.

{\bf Limit point / isolated point.}

$x$ is a limit point of $S$ if $\forall \eps > 0$, $\exists y \neq x$ such that $y \in B_{\varepsilon}(x) \cap S$.

$x$ is an isolated point of $S$ if $\exists \varepsilon > 0$, s.t. $B_{\varepsilon} (x) \cap S= \left\{ x \right\}$.

{\bf Open / closed sets.}

$S$ is open if every $x \in S$ is interior to $S$.

$S$ is closed if it contains all its adherent (or limit) points.

{\bf Perfect / bounded / dense sets.}

$S$ is perfect if it closed and contains no isolated points.

$S$ is bounded if $\exists p \in X$ and $M \geq 0$ so that $\forall x \in S, d(x, p) \leq M$.

$S$ is dense if $\forall x \in X$, $x$ is adherent to $S$.

{\bf Cover of a set.}

A cover of a set is a set of subsets whose union equals the original set.  If $C = \left\{ U_{\alpha}; \alpha \in A \right\}$ is an indexed family of sets $U_{\alpha}$ then $C$ is a cover of $X$ if
\begin{align*}
  X \subseteq \bigcup_{\alpha in A} U_{\alpha}
\end{align*}

{\bf Compactness.}


A subset $K \subseteq X$ of a topological space is called compact if $\forall \mathcal{G} \subseteq T$, with 
\[
  K \subseteq \bigcup \mathcal{G},
  \]
  $\exists$ a finite $\mathcal{G}'$ such that $K \subseteq \bigcup \mathcal{G}'$.

  That is, each cover of $K$ has a finite subcover.

  {\bf Prove that compact sets are closed.}

  {\bf Bolzano-Weierstrass Theorem.}

  {\bf Heine-Borel Theorem.}

  {\bf Inverse powers.}

  If $\alpha > 0$ and $p \geq 1$ is an integer, there exists a unique $\beta > 0$ such tha $ \beta^p = \alpha$.

  \begin{proof}
    Uniqueness is easy once existence is shown.  Assume $0 < \alpha < 1$, (since $\alpha = 1$ is easy, and for $\alpha > 1$, just take $\left( \frac{1}{\beta} \right)^n = \frac{1}{\alpha}$ \todo{check this detail}.

      Consider two sequences $(a_n)_n$ and $(s_n)_n$ with

      \begin{align*}
        a_n &= \max \left\{ k \in \mathbb{Z}: \left( \frac{k}{2^n} \right)^p \leq \alpha \right\}.
      \end{align*}
      and $s_n = \frac{a_n}{2^n}$.  We define these because $(s_n)_n$ consists of better binary approximations to $\alpha^{1/p}$.  Need to check that $a_n$ is well defined, but that is not too difficult.

      Claim: $(s_n)_n$ is increasing and bounded above. It is bounded above by 1, since $s_n > 1$ would imply $s_n^p = (a_n / 2^n)^{p} > 1 > \alpha$, which contradicts the definition of $a_n$.  To see that is $(s_n)$ is increasing is straightforward.  Thus, $(s_n)_n$ converges to some limit $\beta$.

      Finally, we will show that $(s_n^p)_n \to \alpha$.  This will show that $\beta$ satisfies $\beta^{p} = \alpha$.  Note that
      \begin{align*}
        \left( \frac{a_n}{2^n} \right)^p \leq \alpha < \left( \frac{a_n+1}{2^n} \right)^{p}
      \end{align*}

      Thus, it suffices to check that the difference between the left and right hand sides approaches zero as $n \to \infty$.  Note that
      \begin{align*}
        (a_n + 1)^{p} - a_n^p = \sum_{k=0}^{p-1} a_n^k,
      \end{align*}
      so since $a_n \leq 2^n$, we have $a_n^k \leq 2^{nk} \leq 2^{n(p-1)}$, when $k = 0, \dots, p-!$.

      In particular,
      \begin{align*}
        \left( \frac{a_n+1}{2^n} \right)^p - \left( \frac{a_n}{2^n} \right)^p \leq \frac{p \cdot 2^{n(p-1)}}{2^{np}} = \frac{p}{2^n}.
      \end{align*}

      Thus, the left hand quantity $\to 0$ as $n \to \infty$.
  \end{proof}

\end{document}
