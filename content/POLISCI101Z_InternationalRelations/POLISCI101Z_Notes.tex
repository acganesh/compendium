\documentclass{article}

\usepackage{adi}

\newcommand{\mat}[1]{\begin{pmatrix}#1\end{pmatrix}}

\title{POLISCI 101Z: Introduction to International Relations}
\author{Adithya Ganesh}

\begin{document}

\maketitle

\tableofcontents

\section{Lecture 1: 6-24-19}

This lecture will focus on the main themes typically discussed in international relations.

Example headline dealing with Iran deal - Iran would agree to not work on nuclear weapons, and US would remove economic sanctions.  Trump decided to withdraw from the deal.

Another example headline - ``EU Decarbonization Plan for 2050 Collapses After Polish Veto.''  Recall the Paris Climate Accord, in which {\ldots}

Headline 3 - ``WTO warns of rising trade barriers ahead of G20 summit.''

Headline 4 - ``UN Report: Record number of South Sudanese face critical lack of food.''  Some 60\% of the Sudanese population are at risk for not having enough food (it's about 7 million people).

As we study these topics, we will think like scientists and ethicists.

Key idea: the problem of international anarchy.  There is no common power that performs the standard functions of a domestic government.

Example functions that a world government performs:

\begin{itemize}
  \item Preserve peace.
  \item Protect environment.
  \item Regulate economy.
  \item Redistribute income.
\end{itemize}

This course examines the causes of (and solutions to) four international problems.

\begin{itemize}
  \item War.
  \item Environment.
  \item Trade.
  \item Poverty.
\end{itemize}

\subsection{Unit 1 - War}

Some recent wars:

\begin{itemize}
  \item War that arose due to US invasion of Iraq in 2003.
  \item India-Pakistan war in 1999.  Over the Cargill sector in Kashmir.
  \item Civil war in Libya, which led to the overthrow of the Libyan regime.
  \item Russian tanks rolling into the invasion of Ukraine.
\end{itemize}

Major wars of the twentieth century:

\begin{itemize}
  \item Assassination of archduke Franz Ferdinand by a Serbian nationalist; many viewed as the trigger for World War 1.
  \item WW2: 15 million soldiers died.
  \item Korean War: 5 million people died.
\end{itemize}

If you ask during what periods major powers were fighting against each other, this is a rarity (despite the significance of the previous inter-state conflicts).

Key questions:

\begin{itemize}
  \item Why does war occur?
  \item Are some countries war-prone?
  \item How can leaders prevent war?
\end{itemize}

Ethical question:

\begin{itemize}
  \item Under what condition might war be justified?
\end{itemize}

\subsection{Unit 2 - Environment}

CO2 production is growing, which is pretty worrisome.  But some problems appear to be getting better - CFC production has fallen a lot since 1987, when the Montreal Protocol was inroduced.

Key question:
\begin{itemize}
  \item When do countries enter (and honor) environment agreements?
  \item What ethical issues should enter the debate?
\end{itemize}

\subsection{Unit 3 - International Trade}

Many people believe in the advantages of free trade.  But not everyone agrees (some people think socialism is better).

There is a global trend toward freer trade.  Average traffics across time have declined since 1985.

Note - agriculture is a major point of contention in trade debates - since countries don't want their farmers to have to deal with foreign competition.

\subsection{Unit 4 - Poverty and Capital Flows}

Globally, there are 746 million people in extreme poverty (in 2013).

Some key questions:

\begin{itemize}
  \item Why do countries give aid, and to whom?
  \item Is the IMF necessary, and is it driven by poltical concerns?
  \item What obligation do rich nations have to poorer ones?
\end{itemize}

Reading load is about 100 pages / week.  Lectures reinforce and supplement the readings.  Attend lectures (M, T, W only), review slides (posted after lectures).

Section timeline: register on Canvas.  Sections begin this week.


\section{Lecture 2: 6-25-19}

Recall that international anarchy means that there isn't a centralized government to manage the world; this creates issues that we may take for granted in a domestic context.

\subsection{Democracy and War}

Is there a connection between democracy and war?  If so, what is the definition.

Quotes from world leaders:

\begin{itemize}
  \item ``Democracies don't attack each other\ldots Ultimately the best strategy to insure our security and to build a durable peace is to support the advance of democracy everywhere.'' (1994 State of the Union Address, Bill Clinton).
  \item ``The reason why I'm so strong on democracy is democracies don't go to war with each other.  And the reason why is the people of most oscieties don't like war, and they understand what war means.'' (George Bush, Nov. 2004).
  \item ``\ldots America has never fought a war against a democracy, and our cloests friends are governments that protect the rights of their citizens.'' (2009 Nobel Acceptance Speech, Barack Obama).
\end{itemize}

Plans for next 3 lectures:

\begin{itemize}
  \item June 25: Theories about democracy and war
  \item June 26: How to evaluate evidence
  \item July 1: Evidence about democracy and war
\end{itemize}

Casual hypotheses have three parts:

\begin{itemize}
  \item Dependent variable
  \item Independent variable
  \item Connecting logic
\end{itemize}

Clinton's dependent variable is {\it peace}, and the independent variable is whether the country is a democracy.

Hypothesis should be

\begin{itemize}
  \item General (eliminate the proper nouns)
  \item Falsifiable (could be proven wrong)
\end{itemize}

We will consider two types of mechanics that could connect democracy to peace.

\begin{itemize}
  \item Structural
  \item Normative
\end{itemize}

Structural mechanisms:

\begin{itemize}
  \item Democracy -> constraints on the executive
  \item Constraints reduce the propensity for war
\end{itemize}

Democratic structures could contribute to peace by:

\begin{itemize}
  \item Empowering voters
  \item Delaying mobilization
  \item Conveying information
\end{itemize}

Empower voters arguments:

\begin{itemize}
  \item Leaders want to remain in office
  \item Voters will remove belligerent leaders
  \item Thus, democratic leaders have electoral incentives to be peaceful.
  \item Autocratic leaders don't have the same constraints
\end{itemize}

This idea is due to Kant, {\it Perceptual Peace}, 1795.

Can critique this argument.  1. isn't necessarily true.  2. some voters want war (e.g. Nazi Germany, also Philippines). 4. Some autocrats may be overthrown if they are sufficiently poorly viewed by the people.

Some case studies to look into:

\begin{itemize}
  \item Putin's Russia, invading Ukraine
  \item Vietnam war, when Nixon got elected
  \item Eisenhower's election during the Cold War.
\end{itemize}

There are also electoral autocracies, e.g. Mexico in the 80s.

Note the ``Delay Mobilization'' mechanism.

\begin{itemize}
  \item Democracies have checks and balances (sometimes called veto players).
  \item These checks delay decisions to use force.
  \item Delay reduces the propensity for war \ldots
    \begin{itemize}
      \item Affording time to negotiate
      \item Reducing the risk of surprise
    \end{itemize}
\end{itemize}

Two critiques: democracy may not lead to delay; and delay may not reduce the risk of war.

{\bf ``Convey Information'' mechanism.}

\begin{itemize}
  \item Democracy increases transparency about intentions and capabilities.
  \item Democracy can increase credibility. 
  \item By raising transparency and credibility, democracy prevents misperceptions that could lead to war.
\end{itemize}

Tomorrow, we'll discuss {\bf normative mechanisms.}

\section{Lecture 3: 6-26-19}

In this lecture, we'll talk about normative mechanisms that could connect democracy with peace.

{\bf Normative Mechanisms.}

\begin{itemize}
  \item At home: Democratic leaders solve disputes peacefully. Autocratic leaders use violence.
  \item Abroad: leaders ``externalize'' domestic norms; they apply the same norms they use at home.
\end{itemize}

{\bf Some terminology.}

\begin{itemize}
  \item Suppose there are two types of polticial regimes.
  \item There are three kinds of dyads: DD, DA, AA.
\end{itemize}

Version 1: ``unconditional externalization.'' we expect that DD is peaceful, AA is war-prone, DA is typically peaceful.

Vesrion 2: conditional externalization.

\begin{itemize}
  \item DD is peaceful.
  \item DA is slightly less peaceful,
  \item And AA is war-prone.
\end{itemize}

Version 3: democratic crusade. Democratic states try to export their norms.
\begin{itemize}
  \item DD: peaceful.
  \item AA: slightly less peaceful.
  \item DA: war-prone.
\end{itemize}

Nuances:

\begin{itemize}
  \item Democratic norms take time to develop.
  \item Democratic norms are not upheld everywhere.
\end{itemize}

When we test theories, we have three steps.

\begin{itemize}
  \item Collect data.
  \item Describe the data.
  \item Analyze the relationships between variables.
\end{itemize}

Ideally, we'd like to run experiments, but often running experiments is difficult.  There are a couple of strategies to use when we can't run experiments:

\begin{itemize}
  \item Select all known cases (may be experiments)
  \item Take a random sample (may be inefficient?)
  \item Choose extreme values of the IV (e.g. Turkmenistan is not democratic).
\end{itemize}

Some things to avoid:

\begin{itemize}
  \item Choosing only one extreme of the IV.
  \item Selecting based on a value of the DV.
  \item Picking cases that ``prove'' your point.
\end{itemize}

Steps 2: describe the data.

There are some simple things:

\begin{itemize}
  \item Find the range.
  \item Compute the mean.
\end{itemize}

Approval of Bush on Iraq graph.

Important statistic: MoE refers to the variance in the result that could emerge based on repeated sampling.

There are a number of ways to analyze relationships between variables.

\begin{itemize}
  \item Cross-tabulation (just taking two variables, and arraying them in a table).
  \item Scatterplot
  \item Regression
\end{itemize}

Nuclear weapons experiment, (Herrmann, Tetlock and Visser, APSR 1999).

They said: think about a country that has savagely attacked its neighbor, a long-time friend of the United States.  The attacker \ldots

\begin{itemize}
  \item Version 1: has no nuclear weapons.
  \item Version 2: has nuclear weapons that give it the capacity to kill millions of people in a single airstrike.
\end{itemize}

Interestingly, more people wanted to use force when the country had nuclear weapons.

Interesting case study - think critically about the quality of the data.  The ``Safe Celebration Study'' (Sept 2004) - do they tailgate safely or not?

\begin{itemize}
  \item Surveyed 986 college students
  \item 9/10 tailgate safely
\end{itemize}

Some concerns\ldots

\begin{itemize}
  \item Funded by Anheuser-Busch
  \item Relies on self-reported behavior
  \item Excluded students under 21
\end{itemize}

\section{Lecture 5: 7-1-19}

The focus of today: focusing on evidence about democracy and war.

Typically, when analyzing theories, we follow three steps:

\begin{enumerate}
  \item Collecting data
  \item Describing data
  \item Analyze relationships.
\end{enumerate}

Recall from before, the dependent variable is war vs. peace, and the independent variable is regime type, in this case democracy.

How do we define war?

Russett defines wars as ``large scale institutional violence.''  Typically, when we talk about war, we refer to wars between sovereign states.

\begin{itemize}
  \item Before WWI, territory was sovereign if it received diplomatic missions from UK and France.
  \item After WWI, could also demonstrate sovereignty by being a member of League of Nations or the UN.
\end{itemize}

Authors exclude colonial, civil, tribal wars. The dataset we will consider is called the Correlates of War dataset.  

\begin{itemize}
  \item Each war must include at least 1000 battle fatalities.  
  \item This excludes:
    \begin{itemize}
      \item Mere declarations
      \item Accidents
      \item Rogue commanders
      \item Unresisted invasions
    \end{itemize}
\end{itemize}

Class survey: how to operationalize democracy:

\begin{itemize}
  \item Free and fair elections; (Fixed periods; 2 or more different parties)
  \item Checks and balances / separation of powers.
  \item Potentially, civilian control of military.
  \item Policy matches what people want.
  \item Civil participation - vote.
  \item Free speech / free religion.
\end{itemize}

How does Russett operationalize democracy?

\begin{enumerate}
  \item Creates an index (polity index), based on
    \begin{itemize}
      \item Political participation, executive recruitment, diffuse power
      \item Excludes civil and economic liberties.
    \end{itemize}
  \item Divides the index into segments $(-100, -25)$ autocratic; $(-25, 30)$ is anocratic, $(30, 100)$ is democratic.
  \item Requires stability ($\geq 30$ for 3 years).
\end{enumerate}

How do Farber and Gowa operationalize democracy?  Two variables, democ / autoc.

\begin{itemize}
  \item If democ score $\geq 6$, then it's a democracy.
  \item If autoc score $\geq 5$, then it's an autocracy.
  \item Otherwise anocratic.
\end{itemize}

They omit interruptions, transitions, and interregnums.

After operationalizing variables, the authors obtained samples.  They collected dyad-years (pairs of countries, corresponding to particular years).

The authors focused on certain dyads and years.

Russett focused on:

\begin{itemize}
  \item 1946 - 1986
  \item Only ``politically relevant'' dyads.
\end{itemize}

Farber and Gowa focused on:

\begin{itemize}
  \item 1816 - 1980.
  \item Execept WWI and WWII.
  \item All dyads, not just politically relevant ones.
\end{itemize}

First, note that war is a rare event.  In Russett's data, of 29081 dyad-years, he finds war in 32.

Scond, note that democracy is rare, too.

\begin{itemize}
  \item Before 1914, only 16\% of countries were democratic.
  \item 1914-1945, 38\% of countries were democratic.
\end{itemize}

Are DD dyads less likely to fight?

In Russett's data, 32 instances of nondemocratic dyad wars.  There were 0 democratic dyad wars.

If democracy were unrelated to war, how many wars would we expect in each type of dyad?

\begin{itemize}
  \item Null hypothesis, war is unrelated to a countries state as a democracy.
  \item Under the null hypothesis, the rate of war overall should be equivalent to the rate of war in the democratic dyads.
\end{itemize}

We can use the chi-squared test to determine whether this difference occurred by chance.

\begin{enumerate}
  \item Scalculated chi-squared, which measures how much the table we observed differed from what we expected.
  \item Use the chi-squared to find the probability of seeing a difference that large by chance alone.
\end{enumerate}

Formula:
\begin{align*}
  \chi^2 = \sum_{k=1}^{n} \frac{(x_i - m_i)^2}{m_i}.
\end{align*}

Recall Russett's table:


\begin{tabular}{|c|c|c|}
  \hline
  & Democratic & Nondemocratic \\ \hline
  War & 0 & 32 \\ \hline
  Peace & 3878 & 25171 \\ \hline
\end{tabular}

Calculating, we find the that $\chi^2$ statistic is 5. To find out how significant this is, use a $\chi^2$ table or a calculated.  If the null were true, mere chance would produce a pattern this strong only 3\% of the time.  Russett concluded: the pattern almost certainly did not arise by chance alone.

\section{Lecture 6: 7-8-19}

{\bf Case study, Israel and Palestine.}

Timeline (focusing on territory).

\begin{itemize}
  \item 1923, British Mandate.
  \item 1947, UN Partition Plan.
  \item 1948: State of Israel declared.
  \item 1948: Arab-Israeli War.
\end{itemize}

Outcome of Arab-Israeli war: Israel gained control of 78\% of the territory.  The armistice line was called the green line.

\begin{itemize}
  \item Israel took Sinai, Gaza, Golan, West Bank, and East Jerusalem.
  \item UN Security Council passed Resolution 242.  Called for Israel to withdraw.
\end{itemize}

Who controls the territories now?

\begin{itemize}
  \item Sinai: Returned to Egypt as part of 1978 Camp David accords.
  \item Gaza: Israel disengaged in 2005, but still controls borders.
  \item Others: Israel still occupies Golan, West Bank, and East Jerusalem.
\end{itemize}

{\bf PLO Goals.}

\begin{itemize}
  \item Sovereign state.
  \item Based on 1967 borders.
  \item Capital in East Jerusalem.
\end{itemize}

{\bf Israeli Goals.}

\begin{itemize}
  \item Jewish state
  \item Democratic state
  \item In Holy Land
\end{itemize}

{\bf Israel wants security.}

Historically, there have been many conflicts In Israel's history from 1948 - 2006.

{\bf There is some common ground.}

\begin{itemize}
  \item Many Palestinians accept Israel, reject vioence.
  \item The Israeli center-left wants a 2-state solution
  \item Most of the international community wants a 2-state solution.
\end{itemize}

Let's apply the framework we've learned.  Perhaps conflict persists because of problems with:

\begin{itemize}
  \item Divisibility
  \item Information
  \item Commitment
\end{itemize}

{\bf Obstacles to dividing territory.}

\begin{itemize}
  \item In theory, territory is divisible.
  \item In practice, division in difficult.
\end{itemize}

{\bf There are Jewish settlements in the West Bank.}

{\bf Water resources in the West Bank.}

{\bf In theory, Jerusalem could be divided.}

\begin{itemize}
  \item Could split or put under international control.
  \item Not a new idea:
    \begin{itemize}
      \item UN partition plan called for corpus separatum
      \item Jordan controlled E Jerusalem 1949 - 1967.
    \end{itemize}
\end{itemize}

{\bf In practice, dividing Jerusalem would be hard.}

\begin{itemize}
  \item Israel says Jerusalem can't be divided (1980 law).
  \item Palestinians say Jerusalem must be their capital.
  \item Holy sites sit together and may be ``indivisible.''
  \item Settlers complicate the issue.
\end{itemize}

{\bf Holy Sites in the Old City.}

{\bf Both sides have reasons to delay.}

\begin{itemize}
  \item For Israel, delay would allow more settlements and walls, changing the de facto division of territory.
  \item For Palestinians, delay would cause demoraphic patterns to shift in their favor.
\end{itemize}

{\bf Democracy is making bargaining difficult.}

\begin{itemize}
  \item We have discussed whether democracy promotoes or impedes peace.
\end{itemize}

{\bf Potential veto players.}

\begin{itemize}
  \item {\it Israel.} In 2009, centrist party (Kadima) won the most seats, but a right-wing coalition took control.  Right-wing Likud won in 2013, 2015, 2019.
  \item {\it Palestinians.} In 2006, Hamas won a majority in the Palestinian parliament, now controls Gaza.
\end{itemize}

{\bf Would the parties keep an agreement?}

\begin{itemize}
  \item A commitment is credible if the actor has an {\it interest} in carrying it out.
  \item Does each side think the other has an interest in carrying out an agreement?
\end{itemize}

{\bf Many Israelis don't trust the Palestinians. Why?}

\begin{itemize}
  \item Palestinians have been attacking from Gaza.
  \item PA might not be able to control extremists.
  \item PA might not be willing to control extremists.
\end{itemize}

{\bf Many Palestinians don't trust Israel.  Why?}

\begin{itemize}
  \item Israel insists new Palestinian state be disarmed.
  \item Israeli settlements are signals of negative intent.
  \item Israel makes regular incurions into ``Area A.''
\end{itemize}

How to ensure commitment?  Some have proposed third-party enforcement.

{\bf But\ldots would third-party promises be credible?}

In our class survey:

\begin{itemize}
  \item If a democracy invaded a neighbor, only 47\% would support US intervention (and only 13\% strongly).
  \item If a civil war broke out, only 46\% would support US intervention (only 7\% ``strongly'').
\end{itemize}

{\bf Would Trump side with Israel?}

\begin{itemize}
  \item US Embassy in Jerusalem
  \item David Friedman, US Ambassador to Israel.
\end{itemize}

{\bf For reflection.}

\begin{itemize}
  \item Why haven't Israel and the Palestinians reached a bargained solution?
  \item What steps would you recommend to promote peace?
\end{itemize}

\section{Lecture 7: 7-9-19}

Has interstate war become obsolete?

On the $x$ axis is the year, and on the $y$ axis is the historical percentage of states involved in war.  Using the same criteria as in the Correlates of War dataset.

{\bf Some impressive zeros since 1946.} 

\begin{itemize}
  \item  There have been no wars between West European countries.  This is pretty unusual.
  \item There have been no wars between developed countries.\footnote{Need to determine definitions of developed country.}
  \item No wars between US and USSR / Russia.
  \item No wars between great powers since 1953.
\end{itemize}

{\bf A puzzle.} If interstate war has been declining, what could explain this trend?

How could we make war less likely?

\begin{itemize}
  \item Increase the costs of war.
  \item Enforce commitments.
  \item Reduce uncertainty.
  \item Eliminate contentious issues.
\end{itemize}

{\bf Since 1945, what might have caused the following:}

\begin{itemize}
  \item Increased the costs of war?
    \begin{itemize}
      \item Nuclear weapons
      \item Common market.
    \end{itemize}
  \item Enforced commitments?
    \begin{itemize}
      \item The UN was formed.
    \end{itemize}
  \item Reduced uncertainty?
    \begin{itemize}
      \item The UN reduce uncertainty.
    \end{itemize}
  \item Eliminated contentious issues?
\end{itemize}

{\bf A few possiblities.}

\begin{itemize}
  \item Democracy.
  \item Trade.
  \item Nuclear weapons.
  \item International organizations.
\end{itemize}

{\bf Democratic peace.}

Statistically, democracy has been spreading; whether you think about it in terms of the number of democracies or the percentage of democratic countries in the world.

{\bf Democracy could contribute to peace by\ldots}

\begin{itemize}
  \item Sensitizing leaders to the costs of war.
  \item Making commitments more credible.
  \item Increasing transparency / information.
\end{itemize}

{\bf Commercial peace.}

\begin{itemize}
  \item World trade has soared since 1950s.
\end{itemize}

Idea that trade contributes to peace:
\begin{itemize}
  \item Montesquieu wrote: ``The natural effect of commerce is to bring peace.''
  \item JS Mill wrote that: ``Trade is the principal guarantee of the peace of the world'' and ``is rapidly rendering war obsolete''
\end{itemize}

{\bf Trade could increase the costs of war.}

\begin{itemize}
  \item Trade is mutually beneficial (due to the theory of competitive advantage; developed by Adam Smith, David Ricardo, and other influential writers).  When two countries trade, it helps both countries.
  \item War disrupts commerce, imposing economic costs.
\end{itemize}

{\bf Trade could reduce uncertainty.}

\begin{itemize}
  \item Countries can use trade sanctions to signal resolve without fighting.
  \item Economic exchange might also contribute to information and mutual understanding.
\end{itemize}

{\bf Trade could eliminate contentious issues.}

\begin{itemize}
  \item In the past, countries conquered territory to gain access to economic resources.
  \item Today, countries trade for those resources.
  \item By reducing the need to take territory, trade reduces one historically common motive for war.
\end{itemize}

{\bf Problems with these arguments}

\begin{itemize}
  \item Unequal trade can be a source of tension.
  \item Trade causes reallocation of resources away from declining sectors to more efficient ones.  But it might create loss of opportunity for people in these declining sectors, and thus violence.
  \item Trade policies can be a source of tension.
\end{itemize}

{\bf Testable hypothesis.}

\begin{itemize}
  \item For each dyad, measure either the volume of trade or how much each depends on trade with the other.
  \item {\bf Hypothesis:} war should be less common in dyads with high trade, than in dyads wiht low trade.
\end{itemize}

Looking at data, trade is correlated with peace.  But wait, the relationship could be spurious; there could be some factor $x$ that leads to both trade and peace.

Examples:
\begin{itemize}
  \item Democracy
  \item Capitalism
  \item Human rights conditions
  \item Affinity / good relations; democracy; alliances; geography; etc.
\end{itemize}

{\bf Wait! Causation could run the other way.}

We hypothesized that peace implies trade; but it could be the other directions.

{\bf Nuclear weapons.}

It could be that the existence of nuclear weapons is contributing to peace.

{\bf Many non-nuclear states are under a ``nuclear umbrella.''}

{\bf Nuclear weapons\ldots}
\begin{itemize}
  \item Increase the cost of war
  \item Reduce uncertainty about capabilities
  \item Increase uncertainty about resolve
\end{itemize}

{\bf Theory: nuclear deterrence.}

\begin{itemize}
  \item Deterrence: a strategy of preventing an attack through the threat of a retaliatory action that would cause unacceptable damage.
  \item Nuclear deterrence: preventing an attack throuhg the threat of nuclear retaliation.
\end{itemize}

{\bf To deter effectively.}

\begin{itemize}
  \item Able to relatiate. This requires survivable ``second strike'' forces.
  \item Willing to reliate. Would leaders actually push the nuclear button?
\end{itemize}

{\bf Evidence for a nuclear peace.}

\begin{itemize}
  \item There has never been a nuclear war.
  \item Nuclear weapons have been used only once, against a non-nuclear state (US-Japan 1945).
  \item Only one war between nuclear states (India-Pakistan 1999), and it was minor.
\end{itemize}

{\bf But non-nuclear dyads have become more peaceful, too.}

{\bf International organizations.}

Today, we have more IOs than ever.

\begin{itemize}
  \item United Nations
  \item Regional organizations, many focused on the topics of security and the maintenance of peace.
\end{itemize}

{\bf International organizations such as the UN can\ldots}

\begin{itemize}
  \item Increase the cost of war, e.g. by banding countries together by fighting a country that is belligerent.
  \item Enforce commitments
\end{itemize}

{\bf But the UN has taken action in relatively feew wars.}

Of 39 interstate wars since 1945\ldots

\begin{itemize}
  \item UN applied military sanctions in only 2 (Korea, Persian Gulf war).
  \item UN applied economics sanctions in 3 (Persian Gulf, Yugoslavia).
\end{itemize}

Of 168 civil wars from 1946-277\ldots

\begin{itemize}
  \item Military sanctions: 2
  \item Economic sanctions: 13.
\end{itemize}

{\bf If the UN were successful in preventing war\ldots}

We might expect:

\begin{itemize}
  \item Most dyads would not erupt into wars.
  \item If a dyad does erupt into war\ldots
    \begin{itemize}
      \item The UN was unable to prevent a war
      \item And might not be able to end the war
      \item So UN involvement might not make sense.
    \end{itemize}
\end{itemize}

{\bf Implication.}

\begin{itemize}
  \item To many, the UN looks like a failure.
  \item But perhaps its success is mostly invisible!
\end{itemize}

\section{Lecture 8: 7-10-19}

{\bf Agenda.}

\begin{itemize}
  \item What causes civil war?
  \item Why are civil wars so prevalent today?
  \item When do civil wars end?
\end{itemize}

{\bf Interestingly, interstate war has become less common, but civil war and insurgency (uprising) have not.}

{\bf What is a civil war?}

\begin{itemize}
  \item A civil war is a violent conflict between the stae and non-state armed groups for political control.
  \item Battle-death threshold
    \begin{itemize}
      \item >1000 battle-deaths total (Fearon and Latin 2003)
      \item 25 battle deaths (UCDP PRIO)
    \end{itemize}
\end{itemize}

{\bf Types of civil wars.}

There are many ways to categorize civil wars, e.g.

\begin{itemize}
  \item Political aims
  \item Ideological cleavages
  \item Foreign intervention
\end{itemize}

{\bf What causes civil wars today?}

We will discuss 2 different classes of explanations for civil war onset:

\begin{itemize}
  \item Motive based explanations
    \begin{itemize}
      \item Grievance
      \item Greed
    \end{itemize}
  \item Opportunity based explanations
\end{itemize}

{\bf Ethnic / Islamist tensions have grown (see graph.)}

{\bf These tensions drive groups to rebel.}

\begin{itemize}
  \item Connecting logic: groups rebel because they ``hate'' the state authority due to differences in identity and cultures.
  \item Hypothesis: Ethnic and religiously-motivated groups are more likely ot cause civil war.
\end{itemize}

{\bf Test: Ethnic hatreds?}

\begin{itemize}
  \item Note that $P(\text{War} | \text{Ethnic Grievance}) = 0.108$.
  \item Note that $P(\text{War} | \text{Non-ethnic grievance}) = 0.073$.
  \item The $p$-value is roughly $0.07$.
\end{itemize}

{\bf Test: religious hatreds.}

\begin{itemize}
  \item $P(\text{War} | \text{Islamist}) = 0.095$
  \item $P(\text{War} | \text{Non-Islamist}) = 0.089$.
\end{itemize}

{\bf Is ``greed'' to blame?}

\begin{itemize}
  \item The state controls lots of ``prizes''
  \item Chronic poverty means few opportunity costs to fighting.
  \item Groups rebel because they can profit from war.
  \item e.g. Blood Diamond case in Sierra Leone.
\end{itemize}

{\bf Lots of group try and fail to rebel.}

\begin{itemize}
  \item Greed and grievance are insufficient to explain civil war onset by themselves.
  \item The state is typically good at destroying (or negotiating with) emerging threats.
  \item Some groups have better opportunities to rebel.
\end{itemize}

{\bf Groups have different opportunities to rebel.}

\begin{itemize}
  \item Groups today are relatively weak; so must be careful in when and how they choose to rebel.
  \item Strategic choices:
    \begin{itemize}
      \item Regular fighting (conventional military engagements)
      \item Insurgent fighting (Guerrilla and terrorism tactics)
    \end{itemize}
\end{itemize}

{\bf Historical civil wars were fought between conventional forces.}

\begin{itemize}
  \item e.g. The Confederacy during the US civil war thought of itself as a fully functional independent government.
\end{itemize}

{\bf Modern civil wars involve insurgency.}

{\bf Why does insurgency matter?}

\begin{itemize}
  \item Insurgency allows rebel group to avoid detection / destruction by the state.
  \item Insurgency still puts pressure on state to meet rebel demands.
  \item Opportunities to conduct insurgency make civil war more likely.
\end{itemize}

{\bf Operationalizing our Hypothesis.}

What conditions make insurgency more likely?

\begin{itemize}
  \item Areas where it's harder to detect rebel groups
  \item Areas where it's harder to destroy rebel groups.
\end{itemize}

{\it Factors that may influence the presence of rebel groups.}

\begin{itemize}
  \item Low population vs. high population density area.
  \item Existence of roads.
  \item Presence of civilian shields (e.g. hospitals, schools).
  \item Low income / low education areas.
  \item Rebel detection tends to be harder in rough terrain.
  \item Rebel destruction is harder in weaker states (e.g. countries below the median GDP tend to be more likely to fight).
\end{itemize}

{\bf What causes civil war?}

\begin{itemize}
  \item Motive and opportunity are not sufficient
  \item War requires:
    \begin{itemize}
      \item Conflicting preferences
      \item Opportunity to negotiate (or fight)
      \item Bargaining failure
    \end{itemize}
\end{itemize}

{\bf Why might opportunities for insurgency cause bargaining failures?}

\begin{itemize}
  \item Rough terrain $\to$ information problems.  e.g. State fails to identify an emerging insurgent threat and miscalculates response.
  \item Weak governance $\to$ commitment problems. e.g. State identifies a threat, but can't credibly commit to acommodate its demands.
\end{itemize}

{\bf Why have civil wars become more common?}

Are conflicts breaking gout more frequently?  Not exactly.  The increase in civil war today is due to the accumulation of ongoing civil conflicts.

{\bf Civil wars last a long time.}


\begin{itemize}
  \item Median duration of conflicts, 1946-2007.
    \begin{itemize}
      \item Civil: 20 months
      \item Interstate: 3.4 months
    \end{itemize}
\end{itemize}

{\bf What are the obstacles to ending civil war?}

{\bf Disarmament dilemma prolongs civil war.}

{\bf Logic of disarmament dilemma.}

\begin{itemize}
  \item Settlement requires rebels to disarm.
  \item But the government can exploit disarmament.
  \item Fearing exploitation, rebels might prefer to keep fighting.
\end{itemize}

{\bf What resolves the disarmament dilemma?}

\begin{itemize}
  \item Power-sharing (e.g. in Iraq)
  \item Independence (but this leads to slippery slope-style problems)
  \item Ceasefire? (e.g. in Syria civil war)
\end{itemize}

But: in nearly all of these cases, fighting resumes.  The core problem is that these are not credible commitments.

{\bf What makes civil war settlements credible?}

\begin{itemize}
  \item Third party enforcement
  \item Why?
    \begin{itemize}
      \item Raises the costs of cheating.
      \item Monitor compliance.
      \item Threaten military force or economic sanctions for non-compliance.
    \end{itemize}
\end{itemize}

{\bf Third-party = peacekeepers.}

The UN frequently serves as a third-party to oversee the enforcement of a civil war settlement.

{\bf Does peacekeeping work?}

If you control for selection effect, then yes.

{\bf Midterm format.}

Midterm is 50 minutes. Two types of questions.
\begin{itemize}
  \item T/F or multiple choice questions (probably 20).
  \item Short answer (choose 4 identification questions out of 6).
  \item Covers reading, lectures, and discussions.
\end{itemize}

Final exam is up to 3 hours.

Review session on Tuesday.

\section{Lecture 9: 7-15-19}

Note that we have a midterm on Wednesday, and there will be a review session.

Two aspects of the ethics of war:

\begin{itemize}
  \item Groups for war (Jus ad bellum)
  \item Conduct in war (Jus in bello)
\end{itemize}

We will describe three different ethical traditions to this question:

\begin{itemize}
  \item Political realism
  \item Consequentialism
    \begin{itemize}
      \item Utilitarianism
      \item Case study: Hiroshima
    \end{itemize}
  \item Non-consequentialism
    \begin{itemize}
      \item Christian Just War
      \item Islamic Just War
    \end{itemize}
\end{itemize}

{\bf Political Realism.}

Realists claim that the real world is a violent world.  And states in IR are like gladiators.

{\bf In this state war\ldots}

\begin{itemize}
  \item ``It is better to seek salvation via the sewer.'' - Bismarck (Have to play the same game, and be brutal in war.)
  \item ``Notions of right and wrong have no place.'' - Hobbes
\end{itemize}

{\bf Example: Melian Dialogue (416 BC).}

This dialogue takes place during the Peloponnesian war.

``the strong do what they can and the weak suffer what they must''

The Athenians crush the Melians.

{\bf Some implications.}

\begin{itemize}
  \item A realist would say that moral behavior is dangerous and irresponsible.
  \item A realist would say that moral rhetoric is hypocritical and pointless.
\end{itemize}

{\bf Utilitarianism and war.}

\begin{itemize}
  \item Consequentialist: the moral value of an action/institution lies in its consequences
  \item Hedonist: pleasure/happiness is the ultimate good
\end{itemize}

{\bf Utilitarian principle.}

An action or policy is morally right if it produces the greatest balance of happiness over unhappiness.

{\bf Applying utilitarianism to nuclear weapons.}

U.S. dropped two nuclear bombs in 1945:

\begin{itemize}
  \item Hiroshima (Aug. 6)
  \item Nagasaki (Aug. 9)
\end{itemize}

{\bf Hundreds of thousands died.}

\begin{itemize}
  \item Hiroshima (in 1945, 140K died; next 5 years, 60K died).
  \item Nagasaki (in 1945, 70K died; next 5 years, 70K died).
\end{itemize}

Total of 340K.

{\bf How might a utilitarian analyze the decision?}

Did it save lives? (Use counterfactual reasoning)

\begin{itemize}
  \item Look at U.S. casualities that may have been saved
  \item Look at the positive good created by ending the war
\end{itemize}

{\bf Claim: the alternative was invasion.}

There were two 

{\bf Invasion would have killed\ldots}

\begin{itemize}
  \item Truman says 500K Americans would have died.
  \item Churchill claims that 1M soldiers would have died.
\end{itemize}

Utilitarians would conclude that the atomic bombing was justified because it saved lives.

Utilitarian counter-arguments.

\begin{itemize}
  \item War might have ended without invasion
  \item Invasion might not have been so deadly
  \item There are more humane ways to use the bomb
\end{itemize}

{\bf War might have ended without invasion.}

Evidence:
\begin{itemize}
  \item Japanese were getting weaker
  \item Soviets were about to tip the scales
  \item U.S. could have relaxed its demands
\end{itemize}

{\bf Invasion might not have been so deadly.}

\begin{itemize}
  \item Martin Bernstein: at most, 46K Americans would have died.  According to Bernstein, the 500K / 1M figures from before were post-hoc justification (and not necessarily true).
\end{itemize}

{\bf More humane way.}

Why not just show them how powerful this weapon is?

But, US worried about:
\begin{itemize}
  \item Failure / embarrassment
  \item Disclosure to enemy (they might have information about military power that could be used against you)
\end{itemize}

But: even after bomb was dropped on Hiroshima, Japan didn't surrender.

{\bf Warning shot.}

Drop a bomb in e.g. Tokyo Bay, A Forest, Mt. Fuji.

{\bf Tactical strike.}

Could we do a tactical strike against a military target?  But, they needed a target 3 miles in diameter.

{\bf Allow evacuation.}

But, U.S. feared

\begin{itemize}
  \item Bomb could fail.
  \item Japan could intercept.
  \item POWs as human shields.
\end{itemize}

{\bf Did dropping the bomb save lives?}

\begin{itemize}
  \item Relative to non-nuclear alternatives?
  \item Relative to other nuclear options?
\end{itemize}

{\bf Divine command and war.}

How authoritative are divine commands?

\begin{itemize}
  \item Texts often ambiguous
  \item They require interpretation
\end{itemize}

{\bf Christian tradition.}

\begin{itemize}
  \item Originated with St. Augustine
  \item Between skepticism and pacifism
\end{itemize}

{\bf Grounds for war.}

\begin{itemize}
  \item Just cause
  \item Last resort
  \item Chance of success (interestingly tied in with utilitarian tradition)
\end{itemize}

{\bf Conduct in war.}

\begin{itemize}
  \item Discrimination (acceptable to kill military, but not civilians.)
  \item Proportional tactics (goal of the war is to solve an injustice; can't use force that isn't proportionate to the injustice).
\end{itemize}

Islamic just war thinking shares similar conclusions.

{\bf Key concepts.}

\begin{itemize}
  \item {\it dar al-Islam} (house of Islam, territory under the control of Islam, the Zone of Peace) versus {\it dar al-harb} (non-Muslim world)
  \item Jihad is striving, toiling for God (somewhat misrepresented in literature / media).
\end{itemize}

{\bf Grounds for war (Islam.)}

\begin{itemize}
  \item Just cause (e.g. there is a debate about whether you can fight a war to propagate Islam)
  \item Last resort 
\end{itemize}

{\bf Jihad against Jews and Crusaders (1998)}

\begin{itemize}
  \item Highly controversial (signed by bin Laden)
  \item Says the US has been occupying holy places, killing Muslims in Iraq.
  \item Calls for military action vs. US allies
\end{itemize}

{\bf Conduct in war.}

\begin{itemize}
  \item Discrimination.  The Quran says ``fight in God's cause against those who wage war against you, but do not transgress God's limits''
  \item Proportional tactics
\end{itemize}

{\bf For reflection: apply ethics to a contemporary issue.}

\begin{itemize}
  \item U.S. counterterrorism policy
  \item The ongoing war in Syria
  \item Preemptive strike vs North Korea
  \item Israeli-Palestinian relations
\end{itemize}

\section{Lecture 10: Midterm review}

{\bf Format.}
\begin{itemize}
  \item Exam: in class (be on time).
  \item True / false questions (20) + ID questions (4 of 6).
  \item Writing utensils.
  \item Exam is closed book, but you get a reading list.
\end{itemize}

{\bf Identification question.}

\begin{itemize}
  \item Clearly and succinctly define the term
  \item Contextualize the term
  \item Consider how it is important to IR
  \item Where possible, cite readings
  \item Write legibly
\end{itemize}

Write for the full time (10 minutes, 2 paragraphs).

{\bf Example: ``Insurgency.''}

\begin{itemize}
  \item Concise definition
  \item Historical example
  \item Use of reading example 
\end{itemize}

Underline key concepts so graders don't miss them.

{\bf Statistics and experiment design.}

{\bf Chi-squared test.}

\begin{itemize}
  \item Statistical test for categorical data
  \item Measures how far the observations deviated from what we expected under the null hypothesis of no relationship
  \item Tells us how likely it is that the observed difference between the categories arose by chance
\end{itemize}

{\bf Reliability vs. validity.}

Used to describe measurements.

{\it Validity.} Does the masure capture the concept?

{\it Reliability.} Would different people's measurements of the concept produce the same results?

We can think of four cases (Valid vs. invalid $\times$ reliable vs. unreliable).

Unreliable but valid (maybe centered around the right thing, but high variance).

{\bf Statistical terminology.}

\begin{itemize}
  \item Independent variable
  \item Dependent variable
  \item Hypothesis
  \item Connecting logic
  \item Spurious Relationship (a relationship that you detect between two variables that might be caused by a confounder. e.g. correlation but not causation)
  \item Operationalization (when you want to measure a concept, and you define a concrete measurable variable)
  \item Null hypothesis
  \item Reliability
  \item Validity
  \item Cross-tabulation
  \item Regress
  \item Scatterplot
  \item Chi-squared test
  \item Empirical evidence
  \item Sampling error (when due to pure chance, your sample differ from the population)
  \item Marging of error
  \item Non-sampling error (e.g. when you non-randomly sample your units, or measure something incorrectly)
  \item Selection effect (e.g. in Fortna reading, when your sample is biased due to some selection reason.  For example, when looking at peacekeeping, sample is necessarily more violent.)
\end{itemize}

{\bf Democratic peace theory.}

{\bf Structural reasons why democracy are more likely to stay peaceful.}

\begin{itemize}
  \item Empower voters, delay mobilization, transparent
  \item Assumptions needed:
    \begin{itemize}
      \item voters dislike war
      \item political leaders care about staying in office
      \item veto players, transparency
    \end{itemize}
\end{itemize}

{\bf Normative.}

\begin{itemize}
  \item States externalize about the norms that that are used to resolve disputes at home
  \item Democracies have solve domestic disputes through peaceful methods and autocracies solve domestic disputes through violent methods
  \item Unconditional vs. conditional
  \item Democratic crusade - export democratic norms to autocracies
\end{itemize}

Recall the readings:

\begin{itemize}
  \item Russett (claims that normative theories of democratic peace are weaker, thinks it is a good thing democracies delay mobilization)
\end{itemize}

{\bf Is Democratic Peace Theory Right?}

\begin{itemize}
  \item Since 1946, only one possible case of two democracies fighting a war (Kargin War 1999, between India and Pakistan)
    \begin{itemize}
      \item Is this by chance?
      \item Are common interests (e.g. opposite to the Soviet Union a better explanation for why democracies have not fought each other?); due to Farber and Gowa (1995).
    \end{itemize}
\end{itemize}

{\bf Bargaining theory.}

The literature / readings are good.  Go back to the section notes (combined section on July 3rd).

{\it Puzzle of War.}

\begin{itemize}
  \item War is costly -> must be deals that both states prefer to war (bargaining range).
  \item Example: Mexican-American war (settlement would have saved casualties, but it fell apart).
  \item Conflicting preferences are not sufficient for war ->
\end{itemize}

Need three things for war to break out:
\begin{itemize}
  \item Conflicting preferences
  \item Opportunity to fight / negotiate (e.g. it's unlikely that Sweden and Nicaragua will fight because of the distance)
  \item Bargaining failure
\end{itemize}

{\bf What is the bargaining range?}

\begin{itemize}
  \item Let's assume that Sate A and state B value some territory at 100.  Both State A and B have a 50 pct chance of winnning the war and would pay $\$ 20$ for fighting.
  \item Calculating the payoff each side can expect from war.

    \begin{align*}
      \text{War value} = \text{Prob. Win} \times \text{Gain of war} - \text{Costs of war}
    \end{align*}

  \item  Bargaining range for each player is the intersection between deals that post prefer to war.
\end{itemize}

{\bf How does the BR change with costs of war?}

\begin{itemize}
  \item The bargaining range expands / shrinks with the costs of war.
  \item e.g. if war becomes more costly, both side would prefer a negotiated settlement.
  \item Bargaining range expands as the costs of war increase.
\end{itemize}

{\bf BR range shifts to the left or right depending on who is more / less likely to win.}

{\bf Causes of bargaining failure.}

\begin{itemize}
  \item Issue indivisbility
  \item Information problems
  \item Commitment problems
    \begin{itemize}
      \item Preventive wars
      \item Preemptive wars
    \end{itemize}
\end{itemize}

{\bf Issue indivisibility.}

\begin{itemize}
  \item Issues that cannot be divided into a range of potential settlements (e.g. King Solomon Baby problem).
\end{itemize}

{\bf Information problems.}

\begin{itemize}
  \item State $A$ sometimes has incomplete information about state B's capabilities / resolve / costs of fighting.
  \item State $B$ has an incentive to misrepresent this informaiton in order to negotiate a better deal for itself.
  \item Private information + incentives to misrepresent increase risk of underestimating opponent's willingness to go to war.
  \item {\it Analogy.} Poker game and bluffing.
  \item Examples: WMDs in war on terror (Iraq War, 2003).
\end{itemize}

{\bf Commitment problem.}

Idea: states may end up in war because one or both sides are not able to commit to abide by the terms of an agreement that would allow them to avoid fighting.

States may have an incentive to defect from their agreement.

Analogy: Stanford Honor Code

There can be many kinds of commitment problems that lead to war, but two common examples are preventive and preemptive wars.

{\bf Preventive wars.}

Wars that occur in the context of large power shifts.

Bargaining range is more favorable to $A$ today, but will be more favorable to $B$ in the future.

Example: U.S. - China wars.  China's economy will eclipse US in 5 years, and thus there's an incentive to fight a war earlier.

Examples: 2003 Iraq War (WMDs), Iran-Iraq War (1981)

{\bf Preemptive wars.}

\begin{itemize}
  \item  These are wars that occur in response to large first-strike advantages.
\end{itemize}

{\it Examples.} Rare, but 1967 Six Day War, and Pearl Harbor are plausible examples.

{\bf Lake (2010/11).}

Uses Bargaining theory to examine outbreak of 2003 Iraq War and evaluates how useful it is.

Critiques following assumptions of bargaining mode:
\begin{itemize}
  \item States are unitary actors.
  \item Bargaining is modeled as a two-player game
  \item Bargaining theory does not include the costs of enforcing a settlement
  \item States are rational actors.
\end{itemize}

{\bf Israel-Palestine case study.}

Israeli Goals:

\begin{itemize}
  \item Jewish state
  \item Democratic state
  \item In the Holy Land
\end{itemize}

PLO Goals:
\begin{itemize}
  \item Sovereign State
  \item Based on 1948 Borders (aka ``Green Line'')
  \item Capital in East Jerusalem
\end{itemize}

{\bf Focus: think about this as an example of the bargaining problem.}

{\bf Issue Indivisbility?}

\begin{itemize}
  \item  In theory, territory is divisible, but in practice it is very difficult.
  \item Complicating issues:
    \begin{itemize}
      \item Israel's Jerusalem law insists that Jerusalem by unified, but Palestinians insist on East Jerusalem as capital of Palestine.
      \item Israel Settlements in West Bank and East Jerusalem
        \begin{itemize}
          \item Changes the de facto division of division of territory
          \item Gives incentive to delay.  
        \end{itemize}
      \item Water Resources in West Bank (very few water resources).
    \end{itemize}
\end{itemize}

{\bf Commitment problems.}

\begin{itemize}
  \item Israelis do not trust Palestinians (Palestinians have been attacking from Gaza).
  \item PA might not be willing or able to control extremists
\end{itemize}

\begin{itemize}
  \item Israel insists Palestinian state to be disaremd
  \item Israeli settlements are signal of negative intent.
\end{itemize}

{\bf Ethics of Warfare.}

Four theories of ethics and war.

{\bf Political realism.}

\begin{itemize}
  \item Thucydides, The Melian Dialogue
\end{itemize}

{\bf Utilitarianism.}

\begin{itemize}
  \item Pick the option that produces the greatest net happiness. (Critique: Holt).
\end{itemize}

{\bf Christian Just War Theory}

\begin{itemize}
  \item Rights to go to war
    \begin{itemize}
      \item Self defense, last resort, chance of success
    \end{itemize}
  \item Rights during war
    \begin{itemize}
      \item Discriminate between combatants and civilians, proportional tactics
    \end{itemize}
\end{itemize}

{\bf Islamic War Theory.}

  \begin{itemize}
    \item Rights to go to war.  Just cause (including propagating Islam, last resort).
    \item Rights during war (Discriminate between combatans and civilians, proportional tactics)
  \end{itemize}

  \section{Lecture 11: 7-22-19}

  In survey data, people say they are quite concerned about environmental issues.  The main reason:

  \begin{itemize}
    \item Pollution is {\it individually} rational.
    \item Coercion could solve the problem, but that's difficult in a condition of anarchy.
  \end{itemize}

  Example: overgrazing (leads to loss of nutrients).

  What happens when too many animals graze?

  \begin{itemize}
    \item Desertification
    \item Soil erosion
    \item Invasive weeds
  \end{itemize}
  Thus, the land eventually becomes unstable.

  There are a couple of cases to consider when thinking about the tragedy of the commons.

  \begin{itemize}
    \item I own the grass (then I'll overgraze)
    \item If you own the grass (then I'll not overgraze)
    \item If the grass is common property (then I'll overgraze).
  \end{itemize}

  Concretely, the pursuit of individual self-interest leads to a collectively bad outcome.

  Identify international environmental problems that have a similar logic.  Several examples, including air pollution, overfishing, aquaculture, etc.

  We can model this broad setting using game theory.  Consider a game between two countries: $A$ and $B$.  Assume the following:

  \begin{itemize}
    \item Each can contribute to reducing pollution.
    \item Both contributing would be better than neither.
    \item But: contributing is costly.
  \end{itemize}

  {\it Numerical example.} Assume:

  \begin{itemize}
    \item If both countries contribute, they produce a public good worth \$4 per country.
    \item If one contributes, the contributor produces a public good worth \$2 per country.
    \item If neither contributes, no public good is produced.
    \item It costs \$3 for each to contribute.
  \end{itemize}

  It follows that:

  \begin{itemize}
    \item If they both contribute, they each get 1.
    \item If only one contributes, one gets 2, and another gets -1.
    \item If neither contributes, they both get 0.
  \end{itemize}

  We can summarize this in a matrix, where rows indicate action of each player, and entries indicate net payoffs per agent.  This is analogous to prisoner's dilemma.

  Importantly, the prisoner's dilemma:
  \begin{itemize}
    \item Does not depend on lack of communication.
    \item Does not arise from uncertainty.
    \item Does depend on a lack of mutual concern.
  \end{itemize}

  Tomorrow, we will discuss potential ways to solve the game.

  \section{Lecture 12: 7-23-19}

  Last time, we emphasized that there isn't a technical solution to the prisoner's dilemma (see the Hardin reading).  There are four potential solutions to this broad class of problems we will discuss:
  \begin{itemize}
    \item Technology
    \item Coercion
    \item Reciprocity
    \item Domestic pressure
  \end{itemize}

  We will also discuss how international agreements can reinforce these mechanisms.

  Science (or nature) could decrease costs and/or increase benefits of contributing.  This could align individual and collective incentives.

  Let's consider the game from yesterday, with a change of payoffs.  What is A's cost of contributing fell from \$3 to \$1.  Then, we might get a table like

  \begin{itemize}
    \item $A, B$ both contribute.  $A$ gets $4-1 = 3$, $B$ gets $4-3 = 1$.
    \item $A$ contributes, $B$ doesn't. $A$ gets $2-1 = 1$, $B$ gets $2-0 = 2$
    \item $A$ does not contribute, $B$ contributes.  $A$ gets $2$, $B$ gets $-1$.
    \item If $A$ and $B$ don't contribute, $A$ gets 0, $B$ gets 0.
  \end{itemize}

  With these new payoffs, $A$ should contribute, and $B$ should not contribute, so we have a new equilibrium.

  Now, if the value of the collective good rose, assume 8 if one contribute, 4 if one contributes.  Then we get the following results:

  \begin{itemize}
    \item $A, B$ both contribute.  $A$ gets 7, $B$ gets 5.
    \item $A$ contributes, $B$ doesn't. $A$ gets 3, $B$ gets 4.
    \item $A$ does not contribute, $B$ contributes.  $A$ gets $4$, $B$ gets $1$.
    \item If $A$ and $B$ don't contribute, $A$ gets 0, $B$ gets 0.
  \end{itemize}

  In this case, the payoff for contributing is always higher (regardless of what the other country does), so the equilibrium is $A$ and $B$ both contributing.

  Example: the ozone layer (blocks harmful UV radiation).  In this setting, science shifted the benefits / costs.  The benefits of action increased as people learned about the dangers of CFCs.  The costs of action decreased as companies developed subtitutes for CFCs.  Political response: 1987 Montreal Protocols: countries pledged that they would phase out over time the use of CFCs with the goal of protecting the ozone layer.

  Kofi Annan stated that the Montreal Protocol is the most successful international agreement that he had ever encountered.

  Broader significance of the Montreal Protocol:

  \begin{itemize}
    \item First treaty to address a global environmental threat
    \item Embodied the principle of ``differentiated responsibilities'' (maybe richer countries can implement these changes more quickly).
    \item Acted without scientific certainty (precautionary principle).
    \item Can strengthen the treaty without formal amendments - (ratcheting provision).
  \end{itemize}

  For more on consts and benefits, see Sprinz and Vaahtoranta.

  Second way to solve international problems: {\it coercion by a strong state.}

  \begin{itemize}
    \item The PD involves a commitment problem: states could promise to contribute, but the promise would not be credible.
    \item A strong state could solve the commitment problem by punishing shirkers and/or rewarding contributors.
  \end{itemize}

  What is an enforcer imposed a \$2 cost on cheaters?  Then the game would go this way:

  \begin{itemize}
    \item $A$ contributes, $B$ contributes, $A$ and $B$ get $1$.
    \item $A$ contributes, $B$ does not contribute, so $A$ gets $2-3 = -1$, $B$ gets $2 - 2 = 0$.
    \item $A$ doesn't contribute, $B$ contributes, $A$ gets $2-2 = 0$, $B$ gets $-1$.
    \item $A$ and $B$ don't contribute; so $A$ gets $-2$, $B$ gets $-2$.
  \end{itemize}

  Some problems with this approach:

  \begin{itemize}
    \item Would the strong state actually punish?  Punishment is costly to both the target of the sanctions, but also the country that is imposing the sanction.
    \item Would other countries join in the punishment?
  \end{itemize}

  {\bf Example.} Whaling moratorium: in 1985, moratorium came into effect.  Whale catch by Japan, Norway, and Iceland has falling drastically since then.

  {\bf Limits of the agreement.} Japan and others have exploited a loophole: countries may whale for ``scientific purposes.''  Now, violators are US allies, e.g. Japan, Norway, Iceland.  Would US really punish these countries?

  But recently, in July 2019, Japan withdrew from the IWC; it promptly resumed commercial whaling.

  {\bf Strategies of reciprocity.} This is the third class of solutions to the problem.

  If the game is played repeatedly - the incentives change.  With repetition, players can use strategies of reciprocity.  You condition your move based on what others do in previous plays of the game.  There are various versions of this:

  \begin{itemize}
    \item ``I will cooperate only as long as you cooperate.''
      \begin{itemize}
        \item e.g. I will limit my fishing if other countries do so.
        \item e.g. I will restrain my use of fossil fuels if other countries do the same.
      \end{itemize}
    \item If leaders care enough about the future, this strategy could sustain cooperation.
  \end{itemize}

  Let's see how reciprocity would work.  Recall the matrix looks like
  \begin{align*}
    \mat{1, 1 & -1, 2 \\ 2, -1 & 0, 0}
    \mat{1, 1 & -1, 2 \\ 2, -1 & 0, 0}
    \mat{1, 1 & -1, 2 \\ 2, -1 & 0, 0}
  \end{align*}

  \begin{itemize}
    \item ``I will cooperate in every period, but if you ever defect on me, I will never copperate with you again.'' (grim trigger strategy).  Then:
      \begin{itemize}
        \item {\it Payoff to copperating: } $1 + 1 + 1 + \dots $.
        \item {\it Payoff to defecting: } $2 + 0 + 0 + \dots $.
      \end{itemize}
  \end{itemize}

  Lessons from the repeated prisoner's dilemma:

  \begin{itemize}
    \item Defection is profitable in the short term.
    \item With strategies of reciprocity, the long-term benefits of ongoing cooperation can outweight the short-term incentive to defect.
  \end{itemize}

  {\bf International agreements can facilitate reciprocity by\ldots}

  \begin{itemize}
    \item Setting clear expectations
    \item Monitoring behavior
    \item Coordinating punishments
  \end{itemize}

  {\bf Mobilize domestic interests.}

  Another way to solve environmental problems: use international greements to mobilize domestic groups.

  \begin{itemize}
    \item Give groups the right to sue in domestic courts.
    \item Create benchmarks for ``naming and shaming''
    \item Foster international linkages among groups.
    \item Change preferencces/beliefs of ordinary citizens.
  \end{itemize}

  But, would citizens support an agreement? Tried to assess whether people would enter into hypothetical agreements (see Bechtel and Scheve (2013)).

  Next assignment: policy memo.  Goal: work with another student to advise the U.S. government about a major problem involving:
  
  \begin{itemize}
    \item Environment (unit 2), or
    \item Trade (unit 3), or
    \item Poverty/aid (unit 4)
  \end{itemize}

  Four-page memo: should have four parts:
  \begin{itemize}
    \item Executive summary
    \item Problem
    \item Solution
    \item Political feasibility
  \end{itemize}

  Deadlines:
  \begin{itemize}
    \item Mon, July 29 at noon (Partner with another student and send names to TAs).
    \item Fri, 8/2 at noon: send TAs two sentences: problem and recommendation.
    \item Wed, 8/14 at 4:30: submit memo.
  \end{itemize}

  On 8/7: no lecture (work with partner on policy memo).

  \section{Lecture 13: 7-24-19}

  Today: we will discuss the ethics of climate change.  Before the Paris agreement, we had the Kyoto Protocol of 1997.

  \begin{itemize}
    \item Annex 1 countries committed to binding reductions in GHG emissions relative to 1990.
    \item Non-annex I countries had no binding commitments
  \end{itemize}

  Kyoto didn't work - no limits on China, India, other developing countries.

  Lack of political support in rich countries:

  \begin{itemize}
    \item US never ratified, then withdrew signature
    \item Canada failed to reach target, withdrew
    \item Japan missed target
  \end{itemize}

  Deepest cuts came from collapse of USSR - because of economic collapse.

  Main features of Paris agreement - 

  \begin{itemize}
    \item Goal: prevent temperature from rising more than $2^{\circ} C$ by 2100 (relative to pre-industrial levels).  Ideally, no more than $1.5^{\circ}$.
    \item Method: intended nationally determined contributions (INDCs).
  \end{itemize}

  Paris attempted to solve the problmes of Kyoto
  \begin{itemize}
    \item All countries submitted INDCs;
    \item Rich countries help LDCs shoulder cost
    \item Regular monitoring and reporting
    \item Commitment to ratchet up.  Initial pledge; and the ability to increase commitment
  \end{itemize}

  Could Paris work?  Factors:
  \begin{itemize}
    \item Technology
      \begin{itemize}
        \item Innovation could decrease costs of contribution
        \item Increasing evidence of harm could spur action
      \end{itemize}
    \item Coercion
      \begin{itemize}
        \item Rich countries will help pay for mitigation
        \item Will they punish countries that cheat?
      \end{itemize}
    \item Reciprocity
      \begin{itemize}
        \item Agreement calls for monitoring and ratcheting
        \item These provisions could facilitate reciprocity
      \end{itemize}
    \item Domestic politics
      \begin{itemize}
        \item Agreement could sway the domestic public
        \item Violations could prompt shaming and lawsuits
      \end{itemize}
  \end{itemize}

  US is\ldots
  \begin{itemize}
    \item Withdrawing from Paris
    \item Increasing oil / gas drilling
    \item Imposing tariffs on solar panel imports
  \end{itemize}

  Question - do we have a moral responsibility to address climate change?

  Utilitarian argument:
  \begin{itemize}
    \item Climate change is bad.
    \item Climate change can be reasonably averted
  \end{itemize}

  Problem 1: scientific uncertainty:
  \begin{itemize}
    \item Negative feedback mechanisms, such as cloud cover
    \item Positive feedback mechanisms, such as methane from thawing permafrost and the ice albedo feedback
  \end{itemize}

  Problem 2: how much to weigh the future?
  \begin{itemize}
    \item Solving climate change requires immediate sacrifices in exchange for future benefits.
    \item How should we weight current versus sfuture payoffs?
    \item Need to compute a discount rate to calculate things effectively.
  \end{itemize}

  Problem 3: contingency
  \begin{itemize}
    \item Assume no single nation can cause or prevent climate change
    \item If others don't act; my country shouldn't act (efforts are futile)
  \end{itemize}

  {\bf Some non-utilitarian approaches to climate justice.}
  \begin{itemize}
    \item Corrective justice
    \item Egalitarian justice
    \item Shared responsbilities
  \end{itemize}

  {\it Corrective justice.} Broadly just means - ``you broke it, you buy it''

  This would imply that US, China, UK should repair climate change.  These are standard objections:

  \begin{itemize}
    \item The harm was unintentional. (reply: if unintentional, countries shouldn't pay punitive damages, but they should still pay compensation)
    \item Current generation shouldn't pay for the sins of previous generations.  (reply: current generations should pay, becaue they are the beneficiaries of exploitation by previous generations)
  \end{itemize}

  Posner - discusses more objections.

  {\it Egalitarian justice.}

  Idea: give each person an equal share of the atmosphere.  Or an equal share of the remaining carbon budget.

  Problem: rich are emitting far more than their equal share.  To implement egalitarian justice, you could:

  \begin{itemize}
    \item Require rich countries to cut emissions
    \item Or: allow emissions trading: rich could buy pollution rights from the poor.  or: cap and trade
  \end{itemize}

  {\it Shared reponsibilities.}

  (from Goodin's article).

  \begin{itemize}
    \item Shared rights
      \begin{itemize}
        \item Sovereign countries have right to exploit resources.  Can't intervene unless there is a transboundary impact
      \end{itemize}
    \item Shared duties
      \begin{itemize}
        \item Shared duty not to pollute
        \item Intervention is supererogatory
      \end{itemize}
    \item Shared responsbilities (the idea he most strongly supports)
      \begin{itemize}
        \item Countries have a duty not to pollute, a duty to ``pick up the slack,'' and also to intervene against others
      \end{itemize}
  \end{itemize}

  Question: no individual can have a discernible effect on global climate change.  Given this fact, do you, personally, have a moral obligation to reduce your emissions of CO2?

  \section{Lecture 14: 7-29-19}

  Free trade refers to unregulated economic activity.  Protectionism is when you impose barriers.

  Many American oppose free trade (65\% argue for more restrictions).

  If you ask economists, 95\% of economists in the US support free trade, while 88\% of economists worldwide support free trade.

  {\bf Classic case for FT.}

  \begin{itemize}
    \item FT increases overall economic welfare (increases the size of the pie).
    \item The pie gets bigger in both countries.  Importantly, both sides gain, even when one country is better at making everything.
  \end{itemize}

  Intuitively - it might be the case that Lebron James is the world's best lawnmower, but it doesn't make sense to get Lebron James to mow everyone's lawn.

  This lays out a case for collaboration, even if you're really smart.

  This is an old argument, dating back to Adam Smith, Ricardo.

  {\bf Consider a simple economic model.}

  That is, suppose there are:
  \begin{itemize}
    \item 2 countries (France and Switzerland)
    \item 2 goods (wine and cheese)
    \item 1 factor of product (labor).
  \end{itemize}
  
  Assumptions:
  \begin{itemize}
    \item Each country has 1 million workers
    \item Production process is linear
  \end{itemize}

  Key concepts:
  \begin{itemize}
    \item Absolute advantage (you can create more product at same amount of time).
    \item Opportunity costs
  \end{itemize}

  We will consider 2 cases:
  \begin{itemize}
    \item Each country has AA in one good
    \item Each country has AA in both goods
  \end{itemize}

  The second case is kind of surprising, which shows that trade is beneficial even when one country has AA in both goods.

  {\bf France has AA in wine, Switzerland has AA in cheese.}

  Suppose:
  \begin{itemize}
    \item France: Wine for 100 labor units, cheese for 50 (red).
    \item Switzerland: Wine for 50 labor units, cheese for 100 (blue).
  \end{itemize}

  You can draw a graph as follows:
%\begin{figure}[ht]
    %\centering
    %\incfig{france-graph}
    %\caption{france-graph}
    %\label{fig:france-graph}
%\end{figure}

Under autarky, each country can consume only what it produces.

What price would be acceptable to both sides?

\begin{itemize}
  \item France: In trade, refuse to pay more than 100w for 50c.
  \item Swiss: In trade, refuse to pay more than 100c for 50w.
\end{itemize}

In particular, there is a wide range of acceptable prices.  Red: range that is acceptable to France; blue: range that is acceptable to Switzerland.

%\begin{figure}[ht]
    %\centering
    %\incfig{bargaining-range}
    %\caption{bargaining-range}
    %\label{fig:bargaining-range}
%\end{figure}

Trade would both to consume more.  Suppose the trading price is 100/100.  Then\ldots, the CPF looks like (it moves outward).

%\begin{figure}[ht]
    %\centering
    %\incfig{cpfs}
    %\caption{cpfs}
    %\label{fig:cpfs}
%\end{figure}

New case: France has AA in both goods.  In this setting, suppose that:

\begin{itemize}
  \item France can produce 100 wine, 50 cheese.
  \item Swiss can produce 10 wine, 20 cheese.
\end{itemize}

It turns out that understanding this case requires us to understand absolute vs. comparative advantage.

\begin{itemize}
  \item Absolute advantage: lower absolute cost of producing $x$.
  \item Comparative advtange: lower opportunity cost of producing $x$.
\end{itemize}

Importantly, Switzerland Has a comparative advantage in cheese.  France has to give up 200 wine to make 100 cheese, while Switzerland has to give up 50 wine to make 100 cheese.

Now, obviously - Switzerland gains from trade.

But also, France gains too:
\begin{itemize}
  \item With trade: France could make 100m wine, then trade 20m wine for 20m cheese.
  \item Result: consume 80m wine, 20m cheese.
\end{itemize}

Without trade:
\begin{itemize}
  \item If France consumed 80m wine, it could only consume 10m cheese.
  \item If France consumed 20m cheese, it could only consume 60m wine.
\end{itemize}

%\begin{figure}[ht]
    %\centering
    %\incfig{case2}
    %\caption{Case 2: CPFs for France shits towards the right, even though they have an AA in both goods}
    %\label{fig:cpfs}
%\end{figure}

Other arguments for free trade:

\begin{itemize}
  \item Trade increases welfare via economies of scale.  In some industries production costs fall as output increases (due to learning, technology, machinery).
  \item Trade increases welfare due to competitions.  Firms will feel compelled to cut costs, enhance their products, and improve their services.
\end{itemize}

Conclusions:

\begin{itemize}
  \item FT increases aggregate welfare, even when one country has an AA in all goods.
  \item Puzzle: why do countries have trade barriers?
\end{itemize}

\section{Lecture 15: 7-30-19}

Even though economists are in favor of free trade, protectionism is widespread.  Why is this?

{\bf Tariffs.} There are two types:

\begin{itemize}
  \item Ad valorem tariff (tax is a \% of the good's value; e.g. sales tax).
  \item Specific tariff (tax is a fixed amount per unit).
\end{itemize}

Interestingly, rich countries have lower tariffs, on average.  Note that:

\begin{itemize}
  \item High income countries (3.6\% avg pct tariff)
  \item Middle income countries (9.2\% avg pct tariff)
  \item Low income countries (12.1\% avg pct tariff).
\end{itemize}

But: tariffs vary within income groups.  E.g. South Korea has 8.9\% tariff, while U.S. has 2.9\% tariff.

Also, tariffs have historically varied over time.

There are also nontariff barriers (NTBs).

\begin{itemize}
  \item Quotas / licenses: quantitative limits on imports.
  \item product standards: block imports that don't meet standards (sanitary, environmental, medical, etc).
\end{itemize}

Subsidies are another type of NTB (they help domestic producers beat foreign competition).

Currency policies can be NTBs.

\begin{itemize}
  \item Restrict access to foreign currency
  \item Devalue your currency (if a country cheapens its currency, foreigners will buy more from that country, and consumers in the country will buy less from foreigners).
\end{itemize}

China as currency manipulator.  e.g. Romney says that companies have shut down and people have lost their jobs because China has not played by the same rules.  Would prevent Chinese consumers from buying stuff from the U.S. and making Chinese 

How could China keep its currency low?

\begin{itemize}
  \item China could print new currency, use it to buy dollars and U.S. debt and hold them as financial assets (China holds $1.1T$ of US debt).
  \item Consequences: supply of Chinese currency rises (lowering its value), demand for the U.S. dollra rises (raising its value).
\end{itemize}

What evidence is there that this is happening?

Before 2010, value of Yuan is completely flat relative to the USD.  Eight year low right before 2019.

The problem of substitutability: there is more than one way to block an import.  Makes it hard to ensure that a country is practicing free trade.

Question: why might protectionism be in the national interest?

\begin{itemize}
  \item National defense
  \item Infant industries
  \item Market power
  \item Industrialization
\end{itemize}

P = protectionism.  Some people say that protectionism means that it is important for national security.

e.g. steel tariffs - administration notes that US should build steel at home.  Similar for aluminum.

Justification for automobile tariffs?  Commerce dept concluded that imports of autos / certain auto parts posed a threat to US national security.

Rebuttal -
\begin{itemize}
  \item What wouldn't qualify?  (it's a slippery slope).
  \item There are alternatives: buying from allies, and stockpiling for emergencies.
\end{itemize}

Infant industries:

\begin{itemize}
  \item Shield young firms from foreign competition until they can succeed on their own
  \item Example - blocking Google / Twitter In China led to Baidu / Weibo
\end{itemize}

Rebuttal:

Might not work:
\begin{itemize}
  \item Can government pick winners?
  \item Can the infact be weaned?
\end{itemize}

There are alternatives:
\begin{itemize}
  \item Let prviate investors support the industry
  \item If you must help an infant, subsidize it
\end{itemize}

Market power:

(Krugman is famous for this)

Talked about how countries that strategically use subsidies to help one country gain at the expense of another.

{\it Example.} Suppose:

\begin{itemize}
  \item 2 regions (US and Europe)
  \item 2 firms (Boeing and Airbus)
  \item Only 1 firm can remain profitable
\end{itemize}

Initial payoffs: numbers in cells are profits, in \$ millions.

\begin{tabular}{|c|c|c|}
  & Airbus Produce & Airbus Abstain \\
  Boeing Produce & (-5, -5) & (100, 0) \\
  Boeing Abstain & (0, 100) & (0, 0).
\end{tabular}

There is no dominant strategy here.  If Boeing produces, Airbus should abstain.  If Boeing abstains, Airbus should produce.

Interestingly: subsidies give Airbus a dominant strategy.  Government might add +10 if they produce, but no plus if they abstain.

\begin{tabular}{|c|c|c|}
  & Airbus Produce & Airbus Abstain \\
  Boeing Produce & (-5, -5+10) & (100, 0) \\
  Boeing Abstain & (0, 100+10) & (0, 0).
\end{tabular}

Knowing this: Boeing will stay out, and Airbus will take the profits.

Rebuttal:

\begin{itemize}
  \item Requires a super wise government
  \item Applies to only a few industries
\end{itemize}

Tomorrow, will discuss argument that protectionism could help LDCs industrialize (shift from primary to secondary products).

\section{Lecture 16: 7-31-19}

Some people argue that protectionism could help LDCs industrialize (shift from primary to secondary products).

\begin{itemize}
  \item Primary: agricultural production
  \item Secondary: industrial production
\end{itemize}

Why is it better to focus on industrial production?

{\it Claim 1.} Industrial goods have better prospects
\begin{itemize}
  \item Engel's law: as income rises, \% spent on food will fall, while \% spent on non-food (industrial) items will rise.
  \item Technology: synthetic substitutes can reduce the demand for primary goods.
\end{itemize}

{\it Claim 2.} Prices of primary goods are more volatile.

Why?
\begin{itemize}
  \item Business cycles in rich countries
  \item Unpredictable weather
\end{itemize}

LDCs used protectionism to address these problems.

Import substitution industrialization:

\begin{itemize}
  \item Latin America, 1930s-1960s
  \item High barriers on final products
  \item Allow inputs to enter freely
\end{itemize}

Did ISI work?

{\bf Domestic policies.}

Key ideas about domestic policies:
\begin{itemize}
  \item Protection actually helps some domestic groups
  \item Their influence dpeends on political institutions.
\end{itemize}

Consider the following model, with these assumptions.
\begin{itemize}
  \item Two products: shirts and cars.
  \item Two factors of production: labor and capital (means machines, factories, etc.)
  \item Two countries with different factor andowments.  One country - lots of labor; another country - lots of capital.
\end{itemize}

As from before, recall that trade leads to specialization.

\begin{itemize}
  \item Under autarky: you make both shirts and cars.
  \item Under free trade: specialize according to comparative advantage
\end{itemize}

Specialization will cause certain industries to expand, others to contract.

e.g. country with a lot of labor will focus on shirts; country with a lot of capital will specialize on capital.

We will consider two theories on who wins domestically.  See Stolper-Samuelson and Ricardo-Viner.

Stolper-Samuelson:
\begin{itemize}
  \item Assumption: both factors of production are highly mobile (labor and capital).  e.g. this means that people who make shirts can move.  And, the equipment that is used to make shirts, can be reconfigured to make BMWs (questionable assumption).
  \item Prediction: trade -> class conflict 
\end{itemize}

In a labor abundant country, labor-intensive industries will grow.  This leads to a shortage of labor in shirt industry, so wages rise.

This leads to surplus capital in shirt industry - so value of capital falls.

In a capital abundant country, capital-intensive industries will grow.  Labor will move from shirts to cars, and capital will move from shirts to capital.

\begin{itemize}
  \item Causes a shortage of capital in auto industry -> value of capital rises.
  \item Causes a surplus of workers, so wages fall.
\end{itemize}

In this setting, FT helps capitalists, hurts workers.

Political implications of SS theory:

\begin{itemize}
  \item In labor-abundant countries:
    \begin{itemize}
      \item Workers should favor free trade
      \item Capitalists should favor protectionism
    \end{itemize}
  \item In capital-abundant countries:
    \begin{itemize}
      \item Capitalists should favor free trade
      \item Workers should favor protectionism
    \end{itemize}
  \item Implies a fight between classes (labor vs capital).
\end{itemize}

Objection to Stolper-Samuelson: assumes that factors of production can be redeployed easily.  True in some cases, not in others.

Ricardo-Viner theory: a different way of thinking about this process.

\begin{itemize}
  \item Assumption: some factors are fully mobile.
  \item Prediction: trade $\to$ conflicts between industries, rather than classes.
\end{itemize}

{\it Example.} Suppose that capital is hard to move.

\begin{itemize}
  \item In a labor abundant country, trade $\to$ redeployment of labor, but some capital remains stuck.  Can't move capital from car to shirt.
  \item How will this affect domestic groups?
    \begin{itemize}
      \item Shirt capitalists will win
      \item Car capitalists will lose
      \item Effect on workers is harder to predict (depends on how much their wages change, etc).
    \end{itemize}
\end{itemize}

The opposite is true in a capital-abundant country.  Car capitalists will win, shirt capitalists will lose.

Political implications of RV:

\begin{itemize}
  \item If capital is immobile, trade will hurt some capitalists while helping others.
  \item If labor is immobile, trade will hurt some workers while helping others.
  \item Thus: battle is between industries, rather than classes.
\end{itemize}

Puzzle:
\begin{itemize}
  \item FT increases economic welfare (overall size of the pie).
  \item So: why don't winners compensate the losers?
\end{itemize}

One approach: trade adjustment assistance.  Try to help workers who lose jobs because of foreign trade. \todo{consider writing about this in policy memo}

Ways to help:
\begin{itemize}
  \item Training
  \item subsidies
  \item healthcare
  \item job search allowance / relocation allowance.
\end{itemize}

Another approach: tax reform.  e.g. Scheve/Slaughter recommend cutting the payroll taxes of workers who earl less than the national median.

Puzzle: why do countries pursue protectionism instead of a combined policy of free trade + trade adjustment?

Briefly, we'll discuss domestic institutions and trade policy.

In U.S., Congress typically sets U.S. tariffs.

\begin{itemize}
  \item Constitution empowers Congress to:
    \begin{itemize}
      \item impose import duties
      \item regulate commerce with foreign nations
    \end{itemize}
  \item president needs 2/3 approval for treaty.
\end{itemize}

Smoot-Hawley tariffs of 1930 - brought protectionism to highest level in US history.

Reciprocal Trade Agreements Act of 1934:

Authorized the president to make reciprocal tariff reductions without congressional arppoval.  Democrats largely supported this, while Republicans argued for protectionism.

Puzzle: why did Congress delegate to the President?

\section{Lecture 20: 8-12-19}

Structure of exam:
\begin{itemize}
  \item 20 true false (10 min)
  \item 6 of 8 ID's (10 min each)
  \item 1 essay (30 min each)
\end{itemize}

Review session is Thursday @ 6:30pm.

Recall the course focuses on four international problems that arise in the context of anarchy.
\begin{itemize}
  \item Unit 1: War
  \item Unit 2: Environment
  \item Unit 3: Trade
  \item Unit 4: Poverty
\end{itemize}

{\bf Puzzle of foreign aid.}
\begin{itemize}
  \item No world government redistributes income
  \item But many governments voluntarily send aid.  Why is this?
\end{itemize}

Reasons countries might give foreign aid: stability, morality, reduce human rights.

{\bf Measuring government aid.} We will talk about ODA (official development assistance).

{\bf Before 1945 countries did not give ODA.}  Instead, rich countries engaged with poor through trade, colonialism, conquest.

ODA emerged after World War II.

\begin{itemize}
  \item In 1948, the US launched the MArshall Plan to rebuild economics of Western Europe.
  \item 18 countries received aid.  More than half went to U.K., France, West Germany.
  \item No aid to Spain (Franco) or Eastern Bloc (Soviets).
\end{itemize}

By the 1960s, most rich democracies had their own aid programs.  Since the 1960s, we can look at a graph of ODA in billions of 2015 US dollars.  As of last year, US gave about \$120B in aid.

ODA may be given:
\begin{itemize}
  \item Directly from the donor government to a recipient country (bilateral aid)
  \item Indirectly via an international organization (multilateral aid).
\end{itemize}

Many international organizations channel aid:

\begin{itemize}
  \item UN (UNICEF, IFA, WHO, UNHCR)
  \item World Bank, etc.
\end{itemize}

Some countries give most of their aud multilaterally.

Sometimes, aid can be tied:

\begin{itemize}
  \item Tied aid must be used to buy oods / services from the donor country.
  \item United aid does not have this requirement.
\end{itemize}

About 40\% of US aid tends to be tied aid.

{\bf Motives for giving aid.}  Could be two main categories of reasons:

\begin{itemize}
  \item Altruistic (give for humanitarian reasons)
  \item Egoistic (give for selfish reasons)
\end{itemize}

To infer leaders, we could study what leaders say, e.g.

\begin{itemize}
  \item Speeches
  \item Interviews
  \item Memoirs
  \item Diaries
  \item Letters
\end{itemize}

But to get more insight, study behavior, e.g. whether they are egoistic / selfish.

If donors were egoistic, they would:

\begin{itemize}
  \item Give little aid
  \item Favor ``important'' countries
  \item Deliver aid bilaterally
  \item Tie their aid
\end{itemize}

On the other hand, if the donors were altruistic, they would:
\begin{itemize}
  \item Have large aid budgets
  \item Help less important countries
  \item Deliver aid multilaterally
  \item Not tie their aid.
\end{itemize}

Let's evaluate US aid according to these criteria (USAID).

\begin{itemize}
  \item Federal budget is \$ 3.8T.  About 1\% of international budget is international affairs (of which a fraction is foreign aid). U.S. gives less than 0.2\% of its national income.
  \item U.S. gives econ aid to strategically important countries.  e.g. Afghanistan, Pakistan, Jordan.  But overall, Sub-Saharan countries are the biggest recipients of U.S. development assistance.
  \item U.S. aid responds to political shifts - fell after the Cold War, but then has greatly risen.  Rose a lot after 9/11.
  \item Interesting, U.S. aid changes a lot based on security council membership.  Interesting research paper: how much is a seat on the security council worth?
  \item U.S. gives less than 30\% multilaterally.
  \item Well, it's tied a lot, so not great (the highest percentage out of any donor countries we are talking about).
\end{itemize}

What about other donors?  Longstanding goal: give 0.7\% of national income.  Most countries have not met this target.  Only countries that have met: Denmark, Sweden, Luxembourg, Normay.

From readings: aid goes disproportionately to:
\begin{itemize}
  \item Former colonies
  \item Military / political allies
  \item Culturally similar countries
\end{itemize}
See (``Who Gets Aid'', short article).

But\ldots most generous donors send aid elsewhere.  e.g. Sweden.

Generous donors don't always give multilaterally, but they do avoid tying their aid.

If foreign aid reflected concerns about thepoor - countries with the most domestic social spending would also give the most ODA.  This is similar to the theme of norm externalization discussed in unit 1.

As Lumdaine predicted - domestic aid is positively related to foreign aid.

\section{Lecture 21: 8-13-19}

Today, we will focus on:

\begin{itemize}
  \item Aid from international organizations.
  \item Effectiveness of foreign aid.
\end{itemize}

Example: international monetary fund.  Need to revisit the 19th century to understand it.

From 1870 - 1914, there was the Gold Standard, which is an exchange rate system.  The idea was:

\begin{itemize}
  \item Each country pegged to gold.
  \item Participants committed to ``convertibility.''
  \item One advantage of this standard - it minimized exchange fluctuations (so exchange range would be constant).
\end{itemize}

The effects of WWI:
\begin{itemize}
  \item Governments printed money to pay for the war
  \item The gold standard collapsed (due to inflation).
  \item It was restored in the 1920s, but\ldots
\end{itemize}

However, the gold standard collapsed again during the great depression.

Common practices during the depression:
\begin{itemize}
  \item Competitive devaluation (to make goods cheaper).
  \item Exchange control.
  \item Protectionist barriers
\end{itemize}

Origins of the IMF: established at Bretton Woods, 1944, to prevent the mistakes of the Depression.  Some inspiration came from White and Keynes.

The Bretton Woods System:
\begin{itemize}
  \item Country A $\to$ USA $\to$ (@ \$35 / oz) gold.
  \item Country B $\to$ USA $\to$ (@ \$35 / oz) gold.
  \item Country C $\to$ USA $\to$ (@ \$35 / oz) gold.
\end{itemize}

\begin{itemize}
  \item That is, all countries peg their exchange rate between its currency and the U.S. dollar.
  \item The U.S. completes the system by establishing an exchange rate with gold.
\end{itemize}

The Bretton Woods System:
\begin{itemize}
  \item Stabilized exchange rates
  \item Constrained monetary policy.
\end{itemize}

IMF played the role of emergency ``creditor.''
\begin{itemize}
  \item Each member paid a quota to IMF (calibrated based on size of economy)
  \item IMF lent to countries in need
  \item But it imposed conditions on borrowers.
\end{itemize}

IMF played the role of ``referree''
\begin{itemize}
  \item Judged whether pegs should be changed
  \item Monitored convertibility of currencies
\end{itemize}

US policies in late 1960s led to
\begin{itemize}
  \item Inflation, overvalued dollar
  \item Shrinking trade surplus
\end{itemize}

Therefore, a country would have two options:
\begin{itemize}
  \item Deflate (politically unpopular)
  \item Devaluate (required others to revalue)
\end{itemize}

Collapse of Bretton Woods
\begin{itemize}
  \item Nixon chose devaluation
  \item Countries moved to floating rates
  \item IMF needed to reinvent itself.
\end{itemize} 

What activities does the IMF undertake today?
\begin{itemize}
  \item Loans
  \item Surveillance (bilateral and global)
  \item Technical assistance (money doctor)
\end{itemize}

Interestingly, during the recent crisis, the biggest borrowers were in Europe!  e.g. Greece, Portugal, Ireland.

Question: are multilateral donors less political?
\begin{itemize}
  \item They include many countries and diverse interests.
  \item But, their management structure favors the US:
    \begin{itemize}
      \item IMF has weighted voting
      \item World Bank usually has a U.S. director
    \end{itemize}
\end{itemize}

About 9 countries form a majority in the IMF (due to weighted voting).

IMF and WB give more to countries that:
\begin{itemize}
  \item Are U.S. military allies
  \item Vote with U.S. in the UN
  \item Serve on the WB governing board
  \item Rotate onto the Security Council
\end{itemize}

For reflection: should aid be given bilaterally or multilaterally?

Rest of lecture - talking about effects of foreign aid.

Argument (from Jeff Sachs) - the poor face a dilemma.
\begin{itemize}
  \item They can't afford to invest.
  \item Without investments, they can't escape poverty.
  \item Thus, the poor are stuck in a poverty trap.
\end{itemize}

What kind of investments could help?
\begin{itemize}
  \item Human investments (health, nutrition, education)
  \item Business: machines
  \item Infrastructure: roads, power, water.
  \item Natural: e.g. land
  \item Knowledge: science / tech.
\end{itemize}

It's argued that aid can provide investment.
\begin{itemize}
  \item Aid to govts $\to$ public investments
  \item Aid ot familiies $\to$ household investments.
  \item Microfinance $\to$ business investments,
\end{itemize}
thereby promoting growth and reducing poverty.  For more on this, see Jeff Sachs' The Development Challenge.

But - aid doesn't always succeed.  e.g. Zambia (source Easterly).

Potential reasons:
\begin{itemize}
  \item Bad inidvidual choices
  \item Bad government policies
  \item Bad political institutions
\end{itemize}



\section{Lecture 22: 8-14-19}

Today, we will talk about ethics and foreign aid.

Trajectory of international poverty over time:
\begin{itemize}
  \item In 2010, about 2.4B people in developing countries were living on less than \$2 per day.
  \item These estimates were adjusted for ``purchasing power parity''.
\end{itemize}

Percentage of world population living in poverty has declined, but total number has stayed the same.
\begin{itemize}
  \item Percentage ha sdeclined, especially in Asia.
  \item But: percent has not changed in Africa.
\end{itemize}

Recent improvement:


\begin{itemize}
  \item Since 2010, economy has improved
  \item 900M lives on less than \$1.90 a day
  \item 2.1B lives on less than \$3.10 a day.
  \item Still: the problems remain enormous.
\end{itemize}

Three ethical perspectives that we can think of:

\begin{itemize}
  \item Utilitarian
  \item Libertarian
  \item Rawlsian
\end{itemize}

Utilitarian case for aid:
\begin{itemize}
  \item We ought to prevent suffering and death if we can do so at low moral cost.
\end{itemize}

According to Singer, it is obligatory to feed victims of famine.

How much ought we give?  Keep giving, until we ``reach the level of marginal utility.''  That is, give until you reach marginal utility (when giving more hurts you more than it helps).

Do you think we have an obligation to\ldots
\begin{itemize}
  \item Prevent suffering if we do so at low moral cost (yes).
  \item Give until we reach the level of marginal utility (this is complex.  I thin it is quite hard to predict the dynamics of ``what helps'').
\end{itemize}

Rebuttal?

\begin{itemize}
  \item Unrealistic: few would give that much
  \item Too impartial: ignores special obligations
  \item Ineffective: aid does not reduce poverty.
  \item Counterproductive: fosters dependency.
  \item My objection: Marginal utility is complex - there is a time dependence.
\end{itemize}

A different utilitarian view: ``Lifeboat Ethics'' (Hardin; wrote about tragedy o fthe commons)

\begin{itemize}
  \item Limited carrying capacity
  \item Any more would sink the boat.
\end{itemize}

The Malthusian dilemma (overshoot and collapse)

Malthus - known for his pessimism about the future of humanity.  Major contributions to economic thought: Principles of population.

Key idea: if you give aid, population might go up, and gradually crash after.

Rebuttals:
\begin{itemize}
  \item Not close to carrying capacity
  \item Technology might change limits
  \item You could promote population control (birth control, etc.)
  \item Why not sacrifice ourselves?
\end{itemize}

Libertarian perspective, part 1

A government that taxes its citizens to provide foreign aid is coercing its people.

Statement from US libertarian party - ``Individuals should not be coerced via taxes into funding a foreign nation or group.''

Libertarian perspective, part 2

Individuals have no obligation to give.
\begin{itemize}
  \item I acquired my property justly
  \item Anything I give is `pure ``charity''
\end{itemize}

Rebuttals:
\begin{itemize}
  \item Assumes property was acquired jtsly
  \item Other values may outweigh freedom.
  \item Need money to exercise liberty.
\end{itemize}

Rawlsian perspective (developed by Beitz)
\begin{itemize}
  \item International distribution of resources is morally arbitrary: a matter of brute luck. (e.g. poverty is concentrated in the topics).
\end{itemize}

In this situation, what would be just?

\begin{itemize}
  \item Beitz imagines international original position (veil of ignorance)
\end{itemize}

Review of unit 4:
\begin{itemize}
  \item Aid from govts.
  \item Aid from IO
    iEffects of aid
  \item Ethics of aid.
\end{itemize}

Tomz - undergrad research program.

\section{Lecture 23: 8-15-19}

Today: talking about Ethics for aid.  Interestingly, citiziens think foreign aid is a high fraction of the budget, but in reality, it is not very high (<1\%).

Question: How do we think about Trump's proposed cuts in the context of a few different ethical frameworks?

Two types of frameworks:
\begin{itemize}
  \item Consequentialist:
    \begin{itemize}
      \item Singer (utilitarian).  Marginal utility - should give until you reach the threshold of marginal utility.

      \item Hardin: lifeboat ethics. Shouldn't have foreign aid, because there is a limited carrying capacity (lifeboat ethics).
    \end{itemize}
  \item Non-consequentialist
    \begin{itemize}
      \item Rawlsian.  Veil of ignorance - probably pro aid.  Distributive justice.  Prefer policy that supports the least advantaged.  Caveat - if aid is going to not least advantaged

      \item Libertarian: Libertarian would agree with cuts.  They believe individuals should decide where the money you own should go.
    \end{itemize}
\end{itemize}



Logistics:
\begin{itemize}
  \item Review session (6:30 - ?)
  \item Extra OH (Sat 12pm - 2pm CoHo)
  \item Final Exam (Sat 7pm - 9pm)
\end{itemize}

\section{Review for midterm}

  Outline:
  \begin{itemize}
    \item Memorize statistical terms (Recall sampling vs. non-sampling error).
    \item Democratic peace theory (structural vs. normative; Russett).
    \item Bargaining theory (bargaining range, and how it changes based on costs of war).
    \item Bargaining failure (issue indivisibility, information problems, commitment problems).
    \item Ethics of warfare (Political realism, Utilitarianism, Christian Just war theory, Islamic war theory).
    \item Case studies
      \begin{itemize}
        \item Kargil War (1999, India / Pakistan, only case of two democracies fighting a war).
        \item Iraq War (2003, WMDs, information problem; preventive wars).
        \item Iran-Iraq wwar (1981, preventive war)
        \item 1967 Six Day war (preemptive war)
        \item Pearl Harbor (preemptive war).
        \item Israel Palestine case study.
          \begin{itemize}
            \item Israel goals (Jewish state; Democratic state; in the Holy Lands).
            \item PLO goals (sovereign state, based on 1948 borders, capital in east Jerusalem).
            \item Think about this as example of bargaining failure.  (Issue indivibislity, commitment, information problems).
          \end{itemize}
      \end{itemize}
    \item Reading overivews
      \begin{itemize}
        \item Farber and Gowa
        \item Fortna (selection effect in peacekeeping)
        \item Russett (democratic peace theory)
        \item Lake (use bargaining theory to examine 2003 Iraq War).
      \end{itemize}
  \end{itemize}

  {\bf Practice A.}

  (Raw reading list).

  \begin{itemize}
    \item Hoover and Donovan, The Elements of Social Scientific Thinking
    \item Russett, ``Democratic Norms and Culture?''
    \item Russett, ``The Fact of Democratic Peace''
    \item Farber and Gowa, ``Polities and Peace''
    \item Frieden, Lake, Schultz, {\it World Politics: Interests, Interactions, Institutions}
    \item Lake, ``Two Cheers for Bargaining Theory'' (Iraq War)
    \item Beauchamp, ``Everything you need to know about Israel-Palestine''
    \item Council on Foreign Relations, ``Crisis Guide: The Israeli-Palestinian Conflict''
    \item Mueller, ``War has almost ceased to exist: an assessment''
    \item Pinker, ``Violence vnaquished''
    \item Fazal, ``The reports of war's demise have been exaggerated
    \item Fearon and Laitin, ''Ethnicity, Insurgency, and Civil War``
    \item Walter, ''The Critical Barrier to Civil War Settlement``
    \item Fortna, ''Does Peacekeeping Keep Peace?``
    \item Thucydides, ''The Melian Dialogue``
    \item Holt, ''Morality, Reduced to Arithmetic``
    \item Crawford, ''Just War Theory and the U.S. Counterterror War``
    \item Cornell, ''Jihad: Islam's Struggle for Truth``
  \end{itemize}

  (Read list with summaries).

\begin{itemize}
  \item Hoover and Donovan, The Elements of Social Scientific Thinking.

    Defines various terms, and experiment design in social science. Regression, Pearson's r, Sample bias, selection, etc.

    \item Russett, ``Democratic Norms and Culture?''

      Notes that there are structural and normative reasons that democracies are more peaceful overall.

      Three structural reasons:

      \begin{itemize}
        \item Empowering voters
        \item Delay mobilization
        \item Convey information
      \end{itemize}

      Normative models:
      \begin{itemize}
        \item Unconditional externalization
        \item Conditional externalization
        \item Democratic crusade
      \end{itemize}

    \item Russett, ``The Fact of Democratic Peace''

      (see above). Did analysis on dyads.  Russett uses $\chi^2$ and concluded that there is a relationship. 

    \item Farber and Gowa, ``Polities and Peace''

      More nuanced analysis on dyad-years.  Probability of war by regime type and time period, number of dyad-years.

      No statistically significant relationship between democracy and war before 1914.


      Peace after 1945 (Cold War) coincides with common interests among a large number of states.

    Farber and Gowa say that the relationship is spurious!  Democracies tend to have common interests, which has led to recent peace.  It's not the case that democracies are inherently More likely to be peaceful.

    \item Frieden, Lake, Schultz, {\it World Politics: Interests, Interactions, Institutions}

      Textbook that broadly describes bargaining range / costs of war.  Incomplete Information.

    \item Lake, ``Two Cheers for Bargaining Theory'' (Iraq War)

      Bargaining theory as one possible explanation of the Iraq War, shows it is an inadequate explanation of the Iraq War.  Two player games vs. multiple actors. States don't quite act rationally.

    \item Beauchamp, ``Everything you need to know about Israel-Palestine''

      1923: British Mandate

      1947: UN Partition Plan

      1948: Arab-Israeli War

      1967: Six Day War

      PLO Goals:

      \begin{itemize}
        \item Sovereign state
        \item Based on 1967 borders
        \item Capital in east Jerusalem
        \item Solution for refugees
      \end{itemize}

      Israeli Goals:

      \begin{itemize}
        \item Jewish State
        \item Democratic state
        \item In Holy Land
      \end{itemize}

      Israel wants security (given history of conflicts).

      Why does conflict occur?  Problems of:

      \begin{itemize}
        \item Divisibility (dividing Jerusalem is hard in practice)
        \item Information
        \item Commitment (history of conflict)
      \end{itemize}

    \item Council on Foreign Relations, ``Crisis Guide: The Israeli-Palestinian Conflict''

      (similar)

    \item Mueller, ``War has almost ceased to exist: an assessment''

      Reasons for peace:

      \begin{itemize}
        \item Democratic peace
        \item Commercial peace
        \item Nuclear peace
        \item IOs
      \end{itemize}

    \item Pinker, ``Violence vanquished''

      Notes the same idea as before.  But also, civil wars are more frequent.  Less bad news: civil wars tend to kill fewer people.

    \item Fazal, ``The reports of war's demise have been exaggerated''
      
      Advances in battlefield medicinea / preventive care.  Military evaluation practices.

    \item Fearon and Laitin, ''Ethnicity, Insurgency, and Civil War``

      Civil war is due to the steady accumulation of conflicts since the 50s / 60s, rather than a sudden change.  Civil wars tend to last a long time.

    \item Walter, ''The Critical Barrier to Civil War Settlement``

      Civil wars rarely end, because of indivisibility, commitment, various issues. \todo{Fill in}

    \item Fortna, ''Does Peacekeeping Keep Peace?``

      Selection effect - peacekeeping does help, but we note that the conflicts that end up requiring peacekeeping tend to be selected for.

    \item Thucydides, ''The Melian Dialogue``

      Athenians crush the Melians.

    \item Holt, ''Morality, Reduced to Arithmetic``

      Criticizes the ``save lives'' argument for dropping the bomb.

    \item Crawford, ''Just War Theory and the U.S. Counterterror War``

      Argues that U.S. Counterterror policy is unethical.

    \item Cornell, ''Jihad: Islam's Struggle for Truth``

      Argues that the term Jihad has been misrepresented to be associated with violent acts, when it actually means ``striving on behalf of God''

  \end{itemize}

  \newpage 
  Subtest A.

  {\it Readings.}

  \begin{itemize}
    \item Russett ``Democratic Norms and Culture''
    \item Farber and Gowa, ``Polities and Peace'' (example of common interest: opposition ot Soviet Union)
    \item Lake, ``Two Cheers for Bargaining Theory''
    \item Beauchamp, Israel-Palestine
    \item Mueller, ``War has almost ceased to exist: an assessment''
    \item Fearon and Laitin, ``Ethnicity, Insurgency, and Civil War''
    \item Walter, ``The Critical Barrier to Civil War Settlement''
  \end{itemize}

  {\it Ideas / examples.}

  \begin{itemize}
    \item Historical case studies
    \item Ethics of warfare (political realism, utilitarianism, Christian vs. Islamic just war theory)
    \item Statistical terms
    \item Preemptive vs. Preventive wars
    \item Causes of bargaining failure
    \item Spurious relationship (relationship 
    \item Validity vs. reliability
    \item Sampling vs. non-sampling error
    \item Civil war - settlement problems + disarmament dilemma (Mexican standoff).  Peacekeeping works well for civil war settlements.
    \item Jus ad bellum vs. jus in bello.
    \item Utilitarian counterarguments to nuclear weapons

    \item Christian tradition vs. Islamic tradition, jus ad bellum vs. jus in bello
  \end{itemize}


  Christian tradition:
  \begin{itemize}
    \item Grounds for war (religion)
      \begin{itemize}
        \item Just case
        \item Last resort
        \item Chance of success
      \end{itemize}
    \item Conduct in war (religion)
      \begin{itemize}
        \item Discrimination
        \item Proportional tactics
      \end{itemize}
    \end{itemize}

    Islamic tradition:
    \begin{itemize}
      \item Dar al-Islam vs. dar al-harb
      \item Just cause (propagate Islam)
      \item Last resort 
    \end{itemize}

    Conduct in war is similar to Christianity
    \begin{itemize}
      \item Discrmination
      \item Proportional tactics
    \end{itemize}


  Example of bargaining failure: Mexican-American war.

  Subtest A, practice solve.

  \begin{itemize}
    \item Russett ``Democratic Norms and Culture''

      Structural and normative models:

      S:
      \begin{itemize}
        \item Delay mobilization
        \item Empower voters
        \item Convey information
      \end{itemize}

      N:
      \begin{itemize}
        \item Conditional \todo{review}
        \item Unconditional
        \item Democratic crusades
      \end{itemize}


    \item Farber and Gowa, ``Polities and Peace'' (example of common interest: opposition ot Soviet Union)

      Argues that democracy / peace is not statistically significant before 1914, and that they can only obtain a correlation after 1945.  Argues that this is due to interests, rather than polities.  Correlates of war dataset.

    \item Lake, ``Two Cheers for Bargaining Theory''

      Uses bargaining theory to analyze Iraq war.  Points out several shortcomings of the model:

      \begin{itemize}
        \item $n$-player games vs. two player games
        \item Assumes rational actors
        \item Imperfect information?
      \end{itemize}
 
    \item Beauchamp, Israel-Palestine
      
      Conflict between Israel and Palestine
      \begin{itemize}
        \item Israel wants to fight PLO over Jerusalem.  Holy site for both.
        \item Bargaining failure:
          \begin{itemize}
            \item Indivisible (Jerusalem for holiness)
            \item Commitment (lack of trust)
            \item Information (lack of information)
          \end{itemize}
      \end{itemize}

    \item Mueller, ``War has almost ceased to exist: an assessment''

      Four key reasons war has almost ceased to exist:

      \begin{itemize}
        \item Democratic peace
        \item Commericial peace
        \item Nuclear peace (deterrence).
        \item IOs
      \end{itemize}

    \item Fearon and Laitin, ``Ethnicity, Insurgency, and Civil War''

      Note that the counts of civil wars have increased since 1945, but this is not due to an increase in conflict rate, but rather a steady accumulation of ongoing conflicts (since civil wars tend to last longer).  The reasons for why are below (see Walter).

    \item Walter, ``The Critical Barrier to Civil War Settlement''

      \begin{itemize}
        \item Issue indivisibility
        \item Disarmament dilemma (commitment problem)
        \item Information problem (insurgents who are hard to track down)
      \end{itemize}
  \end{itemize}

  {\it Ideas / examples.}

  \begin{itemize}
    \item Historical case studies
      \begin{itemize}
        \item Kargill War (India and Pakistan) - the only war between two democratic states since 1945.
        \item Mexican American War (failure of bargaining; outcomes for both parties were much worse than the available deal)
        \item Palestine conflict
        \item Iran-Iraq war (not sure \todo{)}
        \item Iraq war 2003 (information problem)
        \item Hiroshima / Nagasaki (utilitarian argument)
        \item Pearl Harbor (preemptive war)
      \end{itemize}

    \item Ethics of warfare (political realism (Bismarck), utilitarianism, Christian vs. Islamic just war theory)
      
      Realists say war is reality, don't be moral.

      Utilitarians: greatest good for greatest number (Mill).

    \item Preemptive vs. Preventive wars

      Preemptive: fighting for first strike advantage

      Preventive: fighting because it'll prevent later conflict.

    \item Causes of bargaining failure

      Indivisibility, Information problems, commitment problems, 

    \item Spurious relationship

      A apparent but false relationship between two variables that is due to a confounder.

    \item Validity vs. reliability

      Does the measure model the concept: valid? Does the measure stay consist across time: reliable?

    \item Sampling vs. non-sampling error

      Sampling: error due to sample being off just by chance

      Non-sampling error: other forms of error.

    \item Civil war - settlement problems + disarmament dilemma (Mexican standoff).

      Settlement is hard (Walter), because of divisibility, Information (terrain), and commitment (Mexican standoff).


    \item Jus ad bellum vs. jus in bello.

      Jus ad bellum: ethics of going to war

      jus in bello: conduct in war

    \item Utilitarian counterarguments to nuclear weapons

      Are more humane alternatives possible?  Is it possible to just demonstrate the power without killing civilians? \todo{Maybe flesh this out further}

    \item Christian tradition vs. Islamic tradition, jus ad bellum vs. jus in bello

      Christian: Self defense, last resort, chance of success.  Jus in bello: discrimination, and proportionality.

      Islam: Just cause and last resort.  Jus in bello: discrimination and proportionality.

      dar al-Islam (house of Islam), dar al-harb (the world outside of Islam).
  \end{itemize}

  \newpage

  Subtest B.

  {\it Readings.}

  \begin{itemize}
    \item Russett, ``Democratic Peace Theory'' 
    \item Lake, ``Two Cheers for Bargaining Theory''
    \item Beauchamp, Israel-Palestine
    \item Walter, ``The Critical Barrier to Civil War Settlement''
  \end{itemize}

  {\it Ideas / examples.}

  \begin{itemize}
    \item Historical case studies
    \item Preemptive vs. Preventive wars
    \item Causes of bargaining failure
    \item Sampling vs. non-sampling error
    \item Utilitarian counterarguments to nuclear weapons
    \item Originator of democratic peace? 
  \end{itemize}


  \newpage

  Subtest B.

  {\it Readings.}

  \begin{itemize}
    \item Russett, ``Democratic Peace Theory'' 

      Structural norms (voter empowerment, delay mobilization, and information)

    \item Lake, ``Two Cheers for Bargaining Theory''

      Criticizes the bargaining model.  Notes that imperfect information (WMDs), commitment problems (No guarantee of regime change after US and Iraq agree to peace).  Also, $n$-player games, and rational actor assumption.

    \item Beauchamp, Israel-Palestine

      This conflict won't resolve because of indivisibility of Jerusalem and commitment issues.

    \item Walter, ``The Critical Barrier to Civil War Settlement''

      Information problems (insurgency), commitment problems (disarmament dilemma), and indivisibility.


  \end{itemize}

  {\it Ideas / examples.}

  \begin{itemize}
    \item Historical case studies
    \item Preemptive vs. Preventive wars

      Preemptive: first strike advantage.

      Preventive war (U.S. China war).  2003 Iraq War.

    \item Sampling vs. non-sampling error

      Sampling: when due to pure chance, sample differs from the population.  Non-sampling error: when you randomly sample units or measure something incorrectly.

    \item Utilitarian counterarguments to nuclear weapons

      More humane ways to conduct. Warning shot.  Bomb a non-populated area.  Show them the capacity of the weapon. Invasion may not have been that deadly.

    \item Originator of democratic peace? Kant.
  \end{itemize}

  \newpage

  Subtest C.

  \begin{itemize}
    \item Historical case studies

      \begin{itemize}
        \item Kargil War (
        \item Iraq War (2003)
        \item Iran-Iraq war
        \item 1967 Six Day war
        \item Pearl Hearbor
      \end{itemize}
    \item Israel vs. Palestine
    \item preemptive vs. preventive.
  \end{itemize}




  \newpage


  Subtest C - practice solve.

  \begin{itemize}
    \item Historical case studies

      \begin{itemize}
        \item Kargil War (first example of a democracy-democracy conflict, India vs. Pakistan).
        \item Iraq War (2003).  Information problem, preventive war.  Crawford war on terror reading.
        \item Iran-Iraq war (preventive war, Iraq wanted to invade Iran following the Iranian revolution, but failed).
        \item 1967 Six Day war (Preemptive war)
        \item Pearl Hearbor (preemptive war)
      \end{itemize}
    \item Israel vs. Palestine

      Roughly: Jerusalem hard to divide, Israel and Palestine have commitment problems.
    \item preemptive vs. preventive.

      Preventive: war that occurs in the context of a large power shift.  Preemptive: war in which first strike advantage is critical.
  \end{itemize}

  \newpage

  Subtest C - practice solve 2.

  \begin{itemize}
    \item Iran Iraq war: example of preventive war.
    \item Iraq war (2003): preventive war.
    \item Six day war (1967), Pearl Harbor (example of pre-emptive war).
  \end{itemize}


  Subtest D. 
 
  Describe the history of the Israel-Palestine conflict.

  Israel declared a state in 1948, Arab-Israeli war happened.  Israel gained control of 78\% of territory.

  PLO goals:

  \begin{itemize}
    \item Sovereign state
    \item Based on 1967 borders
    \item Wants capital in East Jerusalem
    \item Solution for refugees.
  \end{itemize}

  Israeli Goals
  \begin{itemize}
    \item Jewish state
    \item Democratic state
    \item In Holy Land
    \item Security.
  \end{itemize}


  \begin{itemize}
    \item West bank control by Palestinian authority
    \item Jerusalem, home to holy sites, Islam / Jewish.
    \item Gaza strip controlled by Hamas.
  \end{itemize}

  \newpage

  Subtest D.

  Describe the history of the Israeli Palestine conflict.

  Arab-Israel war happened in 1967.

  PLO goals:
  \begin{itemize}
    \item Sovereign state
    \item 1967 Borders
    \item Wants capital in East jerusalem
  \end{itemize}

  Israeli Goals
  \begin{itemize}
    \item Jewish state
    \item Democratic state
    \item Holy land 
    \item Security
  \end{itemize}

  West bank controlled by Palestinian authority
  Jerusalem, holy sites.
  Gaza strip: controlled by Hamas.




\subsection{Hoover, 11-35}

\begin{itemize}
  \item ``by tinkering with the meanings of concepts, one can play with the foundations of human understanding and social control.''
  \item Variable: a name for something that is thought to influence a particular state of being in something else.
\end{itemize}

\subsection{Hoover, 38 - 46}
  \item Theory - ``a set of related propositions that attempt to explain, and sometimes to predict, a set of events''
  \item Model - ``an implication of greater order and system in a theory''
  \item Paradigm - ``larger frame of understanding, shared by a wider community of scientists''
  \item Laws / axioms (there are few in social science).
  \item Induction: building theory through the accumulation and summation of a variety of inquiries.
  \item Deduction: using the logic of a theory
\end{itemize}

\subsection{Russett, Democratic Norms and Culture}

\begin{itemize}
  \item Democracies are not always peaceful.  It ignores the rally around the flag effect.
  \item ``When a player employing a conditionally cooperative strategy like tot-for-tat is confronted by someone playing a consistently noncooperative strategy, noncooperation dominates.''
\end{itemize}

{\bf The cultural / normative model.}

\begin{itemize}
  \item Violent conflicts between democracies will be rare.
  \item Violent conflicts between nondemocracies, and between democracies and nondemocracies will be more frequent.
\end{itemize}

\subsection{Farber and Gowa}

\begin{itemize}
  \item No statistically significant relations between democracy and war before 1914.
  \item Only after 1945 that the probability of war is significantly lower between democracies than between members of other pairs of states.
  \item Correlates of war dataset.
\end{itemize}

\subsection{Russett 1993 - Factor of Democratic Peace}

\begin{itemize}
  \item Democratically organized 
\end{itemize}


\section{Section 1: 6-27-19}

{\bf Writing polisci.}
\begin{itemize}
  \item Response paper: 2 total, 1 before July 12.
  \item 2 p. 2x-spaced paper / email Iris by Wed 7pm.
\end{itemize}

Possible options for a response paper:
\begin{itemize}
  \item Critique the reading.
  \item Propose alternate explanation to describe the phenomenon.
  \item Arbitrate between different arguments.
  \item Apply to a historical / current case.
\end{itemize}

{\bf Russett (1993).} His main puzzle is:

{\it Q.} What is the relationship between democracy and war?

It's a conditional normative mechanism.

\begin{itemize}
  \item Democratic states have peaceful norms of conflict resolution.
  \item Autocrats have violent norms of conflict resolution.
  \item Democratic states believe autocratic states are untrustworthy.
  \item Democratic state will pursue violent norms against autocrats.
\end{itemize}

\section{Section 2: 7-3-19}

Data analysis assignment: due July 16.

Recall that the Iraq War was a conflict bwetween US and Iraq, 2003 - 2011.  Fought over the belligerent nature of the regime under Saddam Hussein.  Also, the US demanded Iraq to shut down WMD program, but they didn't.  But it turned out that WMD program actually didn't exist.

Why do wars occur? 2 Explanations.

\begin{enumerate}
  \item Behavioral theories of war.
    \begin{itemize}
      \item Assumption: Leaders fallible to cognitive biases.
      \item Leaders misinterpret information aka perceive an action incorrectly
      \item Leaders are irrationally overconfident about capabilities.
    \end{itemize}
  \item Bargaining model of war.
    \begin{enumerate}
      \item Assumption: States are rationalm try to maximize expected payoffs.
      \item If war is costly: there always should be some deal that is less costly than war itself.
      \item Expected payoff of fighting:
        \begin{align*}
          \text{(probability of fighting)} \times \text{(payoff if win)} + \text{(probability of losing)} \times \text{(payoff if lose)} - \text{cost of fighting}
        \end{align*}

        There are some limitations of this model; if the resource is indivisible, then you won't be able to reason about the expected value.
    \end{enumerate}
\end{enumerate}

Recall the class example of bargaining:

\begin{itemize}
  \item Two players, Iris and Zuhad, can split \$100.
  \item If they fight, winner takes all (payoff = 100).
  \item Iris pays \$20 in medical bills, aka cost of fighting = 20; Zuhad pays \$40.
  \item Players have an equal chance of winning pr(win) = 0.5.
\end{itemize}

Expected value for Iris:
\begin{itemize}
  \item $0.5 \times 100 + 0.5 \times 0 - 20 = 30$
\end{itemize}

This means that Iris prefers any deal that gives her at least 30.

Expected value for Zuhad:
\begin{itemize}
  \item $0.5 \times 100 + 0.5 \times 0 - 40 = 10$
\end{itemize}

So we get a bargaining range between $[30, 90]$ that should be acceptable to both parties.

If bargaining range still exists, why do wars happen?

\begin{enumerate}
  \item Issue indivisibility (e.g. Israel-Palestine Conflict)
  \item Information problem
    \begin{itemize}
      \item Poker game problem; people bluff and misrepresent their capabilities
      \item e.g. Berlin crisis
    \end{itemize}
  \item Commitment problems
    \begin{itemize}
      \item Stanford Honor Code (aka can't trust people to do what you want them to do)
      \item Example: Iran Nuclear Deal
    \end{itemize}
\end{enumerate}

{\bf Bargaining Failures.}

\begin{itemize}
  \item Information problem: Hussein had incentives to conceal information to US re: weapons program.
  \item Commitment problem: US-Iraq agreement was reached, no guarantees US wouldn't use power to induce regime change.
\end{itemize}

{\bf Behavioral.}

\begin{itemize}
  \item States not unitary actors - domestic political actors.
  \item Self-delusions and Bush's gut feeling about Hussein.
  \item Inability to estimate costs of war
  \item Cognitive biases, self-delusions, failure to update beliefs with new facts.
\end{itemize}

\section{Section 3: 7-11-19}

{\bf Agenda.}

\begin{itemize}
  \item Data analysis tips.
  \item Recap: end interstate war?
  \item Application: GP Rivarly.
\end{itemize}

Passed out review sheet.  Recall that the response paper must be turned in before next Wednesday.

The structure of the data analysis:

\begin{itemize}
  \item Intro: Preview your results
  \item Connecting logic 
    \begin{itemize}
      \item Reasons / evidence (e.g. empirical support) $\to$ Identify the observable implications of connecting logic.
      \item Analyze hidden assumptions.
      \item Note, since all the data has to do with public opinion data, hypothesis should probably include the word belief.
    \end{itemize}
  \item Results
    \begin{itemize}
      \item Look at Marginal / Conditional probabilities.
      \item $\chi^2$ test.
    \end{itemize}
  \item Analysis
    \begin{itemize}
      \item What do the results mean?
      \item Sampling error (is the sample representative of the world's population distribution?)
      \item Non-sampling error (problems that could occur by chance; e.g. differences between Stanford students and broader population; question wording; could prime the respondent to answer in a certain way).
      \item Example of question wording: asking whether we want intervention in Iraq could be interpreted as asking whether we want to intervene in 2003 Iraq; vs. asking whether we want to intervene in ISIS.
    \end{itemize}
  \item Conclusion.
    \begin{itemize}
      \item Found some difference / no difference in the subpopulations associated with IV vs. not.
    \end{itemize}
\end{itemize}

Recall that there are four reasons we've seen decline in interstate war:

\begin{itemize}
  \item Democratic peace
  \item Commercial peace (increase in globalization)
  \item Nuclear peace
    \begin{itemize}
      \item e.g. India / Pakistan; even though there 
    \end{itemize}
  \item IOs (International Organizations)
\end{itemize}

However, recently - people argue that we have seen a return to great power rivaly.

Two possible adversaries:
\begin{itemize}
  \item Concerns about China
    \begin{itemize}
      \item (e.g. US and China are currently in the midst of a brewing trade war)
      \item Concerns that something will happen in Taiwan; US is obligated to defend Taiwan (see Taiwan American relations Act, etc.)
    \end{itemize}
  \item Concerns about Russia.
    \begin{itemize}
      \item Russia withdrew from a nuclear arms agreement.
      \item Russia embarked on a nuclear modernization program (would increase the size of the arsenal, and the destructive power of the weapons).
      \item Concerns among many that more nuclear weapons don't make the world more stable, but they increase the likelihood of accidents.
    \end{itemize}
\end{itemize}

{\bf Question.} What are the odds?

\begin{itemize}
  \item What are the odds of a US China war; vs. what are the odds of a US Russia war?
\end{itemize}

{\bf Factors for China.}

\begin{itemize}
  \item Nuclear peace
  \item Commerical peace
  \item Democratic vs. nondemocratic peace
  \item Brinkmapship / accident?
\end{itemize}

Important factor: Freedom of Navigation Operations; complicated maritime law.

US economy is currently larger than China; but China's economy will be larger in 5 years.  So it may make sense to wage war earlier (preventive war).

If China / US got into war, it's possible that other countries would be involved.

US is agresssively building the Navy because of vulnerabilities in the Pacific.

The Thuycidides trap (On the fear of a large nation overpowering another one leading to an increased probability of war).

{\bf Factors for Russia.}

\begin{itemize}
  \item Nuclear peace
  \item Historically, US has had more historical conflict with Russia.
  \item How Russia fights war:
    \begin{itemize}
      \item Insurgent fighting (as opposed to regular fighting on a battlefield).  US has less infrastructure to fight an insurgent war.
    \end{itemize}
\end{itemize}

Consider survey data, what is the threat of Russia?  We can analyze survey data.

\begin{itemize}
  \item $IV$: Partisanship.
  \item $DV$: Perceived threat of Russia.
\end{itemize}

If we have a matrix
\begin{align*}
  A = \mat{a & b \\ c & d},
\end{align*}
where the entries are frequencies representing
\begin{itemize}
  \item $a$: Republican; not threat.
  \item $b$: Republican; threat.
  \item $c$: Democrat; not threat.
  \item $d$: Democrat; threat.
\end{itemize}

We can compute various quantities:

\begin{align*}
  Pr(\text{Rus Threat}) &= \frac{b+d}{a+b+c+d} \\
  Pr(\text{Threat | Republican}) &= \frac{b}{a+b} \\
  Pr(\text{Threat | Democrat}) &= \frac{d}{c+d}.
\end{align*}

Given the data, we have
\begin{align*}
  Pr(\text{Rus Threat}) &= \frac{238 + 113}{32 + 72 + 238 + 113} = 0.77 \\
  Pr(\text{Threat | Democrat}) &= \frac{238}{32 + 238} = 0.88 \\
  Pr(\text{Threat | Republican}) &= \frac{113}{72 + 113} = 0.61.
\end{align*}

Even without doing a $\chi^2$ test, we can see that there's a large divergence in opinion.

If we do the test, we find that the $p$ value is less than $0.5$.

Do we think that the poll has sampling error?

Representatives
\begin{itemize}
  \item English only
  \item  Only 869 pepole
  \item Online
\end{itemize}

Question wording:

\begin{itemize}
  \item Scale
  \item ``Imminent''
  \item ``Threat''
\end{itemize}
Questi


\section{Section 4: 7-18-19}

Midterm. If you say things that are wrong; they will take off points.

Should the US use drones?  Pros and cons.

Pros of drones:

\begin{itemize}
  \item Drone strikes make US safer by decimating terrorists.

    upwards of 3500 militants killed.

  \item Drones kill fewer civilians.  PCT of fatatliesi

  \item Drones make US military personnel safer.

    Less room for human error.

  \item Drone strikes are cheaper than engaging in ground / manned aerial combat.

    $5B$ allocated for Drones, only about 1\% of the entire military budget.

  \item Drone strikes are legal under internationa l law.

    Aritcle 51, sel defense. anticipitaroy self defense.

  \item Drone strikes ar elegal under US law.
  \item Drones limit the scope / scale of military action.
  \item Subject to strict review process.
  \item Cannot risk falling behind rest of the world.
  \item Drone pilots have a lower risk of PTSD than pilots of manned aircraft.
  \item Majority of Americans support drone strikes.
\end{itemize}

Cons:

\begin{itemize}
  \item Drone strikes create more terrorists than they kill.
  \item Drone strikes target individuals who may not be  terrorists / combatans.
  \item Kill large numbers of civilians / traumatize populations.
  \item Kill low value targets
  \item Violate internation law.
  \item Secretive, prevent citizens from holding accountable
  \item Violate sovereignty of other countries (without permission).
  \item Allow US to be emotionally disconnected from horrors of war.  May propagate war.
  \item US drone strikes give cover for others to engage in human rights abuses.
  \item Extremely unpopular in affected countries.
  \item Drone operators have stress.
\end{itemize}

Team antidrone.

\begin{itemize}
  \item Lot of accidents happen.
  \item Drone fails to target military leaders In the past.  Scuceeded in killing 14 military leaders of terorrism, but more civilians.
  \item Anti-US sentiment.
  \item Killing military leaders is not helpful in solving the true issue.
\end{itemize}


According to a July 18, 2013 survey by Pew Research, 61\% of Americans supported drone strikes in Pakistan, Yemen, and Somalia.

{\it Core arguments.}

\begin{itemize}
  \item Accidents.  Crawford mention that accidents can make it possible to sidestep responsibility.  Big challenge - how do you develop metrics that quantify the impact of drone strikes (got to measure civilian attitudes / radicalization).
  \item Drones reduce the risk for PTSD.  Anti-drone mentioned the psychological effect impacts civilians at well (terrorizing relative to other military options).
  \item Proportionality. Effective in getting outcomes we want, but comes at a larger cost.  e.g. Haven't had an attack from Al Qaeda since 2001, ISIS hasn't really done anything in a while.
  \item Thinking in analogies (compare to WW2).  Reason that civilian casualities were very high is that firebombing Dresden was a good way to attack Nazis.
\end{itemize}

\section{Section 5: 7-25-19}

Policy memo.

Topic should be:

\begin{itemize}
  \item Related to trade / environment / poverty.
  \item US-focus.
  \item Feasibility - needs to be tractable and you should be discuss why.  Should be sufficiently limited to that it's possible to find research and support.
\end{itemize}

Research:

\begin{itemize}
  \item Scholarly articles, books, think-tanks, NGOs. (e.g. see Heritage / Cato / Brettonwoods).
\end{itemize}

Writing:
\begin{itemize}
  \item Emphasize broadcasting in advance, and be brief.
\end{itemize}

Pieces:
\begin{itemize}
  \item Exec summary
  \item Problem
  \item Solution
  \item Feasibility
  \item Conclusion
\end{itemize}

\section{Section 7: 8-8-19}

Announcement:
\begin{itemize}
  \item 2nd response paper due Wed 8/14 by 7pm
  \item policy memo due 8/14
  \item final exam review 8/15 (6:30 - 7:20)
  \item final exam, Saturday 8/17 (7pm - 10pm)
\end{itemize}

Agenda:
\begin{itemize}
  \item Protectionism / tariffs
  \item US-China trade war
  \item Ethics of trade
\end{itemize}

Recap: free trade would help everyone, but states can't credibly commit to not be protectionists.  Can model this with a prisoner's dilemma.

Two broad models:
\begin{itemize}
  \item Stolper-Samuelson: trade -> class conflict. (asusmptions: labor and capital are highly mobile).
  \item Ricardo-Viner: trade -> conflicts between industries (rather than classes). (assumptions: some factors are fully mobile).
\end{itemize}

Why protect?
\begin{itemize}
  \item National interest
  \item Domestic politics (SS and RV)
  \item International politics -> optimal tariffs
\end{itemize}

Four broad ways to solve the commitment problem:
\begin{itemize}
  \item Change costs / benefits
  \item Third-party / outside actor
  \item Strategies of reciprocity (tit for tat or grim trigger).  Alexrod reading talks about why tit for tat is better.
  \item Domestic pressure.  (Valhontra reading).
\end{itemize}

{\bf US-China trade war.}  It starts with a commitment problem, which states that:

\begin{itemize}
  \item US and China can't commit to FT because tariffs can shift the terms of trade.
    \begin{align*}
      \text{ToT} = \frac{\text{price of exports}}{\text{price of imports}}.
    \end{align*}
    Example: consider the ratio $\frac{\text{price of soybeans}}{\text{price of iPhones}}$.  If numerator increases while denominator decreases, trade becomes more favorable to the U.S.
\end{itemize}

Last week: trade talks in Shanghai.

Why is there a US-China trade war?  Five underlying factors.

\begin{itemize}
  \item Trade deficit (US is importing a lot more than they are exporting; roughly US imports \$500B more than it exports).
  \item China currency manipulation / fair market (WTO accession / manipulation).  China opened up its economy in 1979.  Transformation is huge - almost more than US from 1800 to now.

    One condition to join WTO is that you need a free and open market.  But U.S. says this is not the case.
  \item IP theft: China takes IP property -> joint venture.  In order for a foreign company to enter into Chinese market, they have to form a joint venture (and join with local players).

    CFIUS: needs to review to make sure that China doesn't acquire / merge investment.

  \item 2025 Initiative (Made in China).  Goal is to invest in AI / advanced technologies; creates a national security risk.  (Huawei controversy).
  \item Great power rivalry: Power transition war?  China may surpass the U.S. no later than 2050.
\end{itemize}
\begin{tabular}[h]{|c|c|c|c|}
  \hline
  Solution & Who? & How? & Weakness? \\ \hline
  Change costs & China & nuclear options w.r.t. U.S. debt.  Would drive up interest rates, kill US economy. & backfire: hurt Chinese economy \\ \hline
  Change benefits & US & Voters love free trade.  Trump might change his mind & Confounders driving FT, costs diffused, populist sentiment (which tend to rise when inequality rises).  \\ \hline
  Reciprocity & Both US and China & committing to de-escalate and reduce tariffs & Mutual suspicion (e.g. Chinese administration is distrustful of Trump).  Shadow of the future. \\ \hline
  International organizations & WTO & Arbitrate / initiate dispute settlement mechanism & US wary of WTO arrangement / veto players. Tariffs are really hard to remove (e.g. US and Canada still have a 25\% tariff on motorbikes). \\ \hline
\end{tabular}

\section{Review for final}

Prisoner's dilemma in int'l politics.
\begin{itemize}
  \item Prisoner's dilemma can be a useful representation of many types of problems
\end{itemize}

Reeated prisoner's dilemma
\begin{itemize}
  \item States often engage in these interactions repeatedly
  \item Using strategies of reciprocity (e.g. Grim trigger, tit-for-tat), states can sustain cooperation if the long term benefits from cooperating outweigh the short term benefit a state can get from defecting.  (Recall Axelrod's tournament)
\end{itemize}

Strategies of reciprocity are most likely to work when\ldots
\begin{itemize}
  \item Players value the future
  \item Reward for defecting is small
  \item etc.
\end{itemize}

Also - recall shadow of the future.

Recall trade: key terms (see the slides).

Engel's Law: as a country gets richer, smaller proportion goes to commodities.

General system of preferences (GSP)

WTO is about nondiscrimination / reciprocity.

Stolper-Samuelson asssumes factors of production are mobile across sectors.

Ricard-Viner does not assume factors of production are mobile across sectors.

\section{Review for final 2}

Need to go over all readings.  Everything in the class is fair game.  Essay should be 30 minutes.

ID:
\begin{itemize}
  \item Explain issue
  \item Contextualize
  \item Provide example
  \item Cite readings
  \item Explain significance for IR
\end{itemize}

Game theory:
\begin{itemize}
  \item Prisoner's dilemma (defect); example of a commitment problem.
  \item If opponent is cooperating, you should defect (i.e. defecting is a dominant strategy; so Defect; defect is the Nash equilibrium).
  \item Repeated prisoner's dilemma.  Need to use strategies of reciprocity (e.g. Axelrod's tournament).
  \item Prisoner's dilemma in int'l politics.
    \begin{itemize}
      \item Classic application of PD: tragedy of commons, optimal tariff argument.
      \item The barrier to settlement of civil war
      \item  Commitment problems generally as a cause of war.
    \end{itemize}
\end{itemize}

Repeated PD:
\begin{itemize}
  \item When games are iterated, you need to be cooperative.
\end{itemize}

When does reciprocity not work:
\begin{itemize}
  \item When players value the future
  \item Reward for defecting is small
  \item Punishment for cheating is long and severe.
\end{itemize}

International institutions
\begin{itemize}
  \item Help states use reciprocity to sustain cooperation
    \begin{itemize}
      \item Set clear expectations (e.g. what counts as ``defecting''?)
      \item monitoring behavior (need to know whether defection has occurred).
      \item Coordinating punishments (helps avoid echo chambers)
    \end{itemize}
  \item Examples: GATT / WTO, Paris Climate Agreement, etc.
\end{itemize}

Key ideas in trade:
\begin{itemize}
  \item Free trade
  \item Protectionism
  \item Autarky (when a state is sustainaible without international trade)
  \item Comparative advantage (when individual can produce an activity more efficiently than other activity).
  \item Absolute advantage (when individual produces an activity more efficiently than another group)
  \item Economies of scale (when scale allows you to do things more efficiently; e.g. bulk discounts).
  \item Ad valorem (tax is \% of good's value) vs specific tariff (fixed amount per unit)
  \item Nontariff barriers
    \begin{itemize}
      \item Quotas (limits on imports)
      \item Product standards (need to regulate sanitary / evnrionmental / medical quality of goods).
    \end{itemize}
  \item Infant industries
    \begin{itemize}
      \item New industries - want to encourage them ideally
    \end{itemize}
  \item Strategic trade policy
    \begin{itemize}
      \item Sometimes good to be protectionist, sometimes good to be freer.
    \end{itemize}
  \item LDCs and Industrialization.
    \begin{itemize}
      \item Least developed countries.
      \item Industrialization is good - because of technology, capital, automation, etc.
    \end{itemize}
  \item Stolper-Samuelson.
    \begin{itemize}
      \item Factors of production (labor / capital) are highly mobile.
      \item In a labor abundant country, labor-intensive industries will grow.
      \item Result: trade -> class conflict.
      \item Summary: Factors mobile -> Class conflict.
    \end{itemize}
  \item Ricardo Viner.
    \begin{itemize}
      \item Factor of productions are not mobile 
      \item Some factors are fully mobile.  Then leads to industry conflict.
      \item S: Factors mobile -> Industry conflict.
    \end{itemize}
  \item Trade adjustment assistance
    \begin{itemize}
      \item Reduce damaging impact of imports.
    \end{itemize}
  \item Tax reform
    \begin{itemize}
      \item Generally, reform taxes.
    \end{itemize}
  \item Smoot-Hawley 1930
    \begin{itemize}
      \item Law which implemented protectionist trade policies in the US.
      \item Highest level in US history at time.
    \end{itemize}
  \item Reciprocal trade agreements, 1934
    \begin{itemize}
      \item Make reciprocal tariff reductions without congressional approval.
    \end{itemize}
  \item optimal tariff
    \begin{itemize}
      \item a country that is a large importer of a particular commodity can shift the economic burden of an import tariff from domestic consumers to foreign  (Chicago PR)
    \end{itemize}
  \item GATT: agreement to promot trade by reducing tariffs / quotas.
  \item Nonscirmination: human right s/ trade?
  \item Most-favored nation: granted the most favorable trading terms available by another country
  \item preferential trade agreements
  \item generalized system of preferences (GSP): economic growth bc. duty free.
  \item Escape calsuses: allows escape trade?
  \item WTO dispute process
    \begin{itemize}
      \item once a complaint has been filed in WTO (multilateral dispute resolution).
      \item If informal consultations fail, panel is formed automatically.
      \item Panel findings
    \end{itemize}
  \item Trragedy of commeons.
  \item Kyoto - failed
  \item Paris - reasonable, prevent global temps from rise >2 deg C.
  \item Four broad solutions: CRDT (coercion, reciprocity, domestic, tech).
  \item Foreign aid
    \begin{itemize}
      \item IMF: Vreeland notes that no one entity controls IMF.
      \item Aid can fail - bc bad individual choices.
      \item Bad govt. / political inst.
    \end{itemize}
  \item Ethics
    \begin{itemize}
      \item Realism
      \item Utilitarian
      \item Just war
    \end{itemize}
  \item Ethics and environ
    \begin{itemize}
      \item Utilitarian
      \item Corrective
      \item Egaliatian
      \item Shared respo.
    \end{itemize}
  \item Trade
    \begin{itemize}
      \item Utilitarian
      \item Rawlsian (original position) difference principle.
      \item Kapstein - RAwls
    \end{itemize}
  \item Libertarian - maximize freedom.
  \item Hardin - lifeboat ethics
\end{itemize}

Reading review:

\begin{itemize}
  \item Easterly: utopian nightmare
    \begin{itemize}
      \item Helping prolonbgs the true nightmare.  Have not gotten around to the most needy countries.
    \end{itemize}
  \item Singer: utilitarian
  \item Hardin - ethics
    \begin{itemize}
      \item Lifeboat ethics relates to population dynamics - since there's a carrying capacity, you'll overshoot.
    \end{itemize}
  \item Armstrong: distributive justice.
    \begin{itemize}
      \item Beitz builds on Rawls
    \end{itemize}
  \item Sweatshops, NY Times.
  \item Kapstein: Rawls, globalization.
  \item McGee: rights.  Trade shouldn't violate property, contract, or association rights.
  \item Beitz: international DP (inequalities should be arranged internationally so they benefit the least advantaged). 
  \item Goodin - shared responsibilities.  
  \item Just war
    \begin{itemize}
      \item Christian / Islamic
      \item Crawford / Cornell
      \item Holt: Morality reduced to arithmeitc.
    \end{itemize}
\end{itemize}

\end{document}
